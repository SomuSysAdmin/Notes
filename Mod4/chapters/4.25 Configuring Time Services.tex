\chapter{Configuring Time Services}

\section{Understanding Time on Linux}
When the system starts up, the hardware clock is reached. From this the system time is set. The system time is the time in software maintained by the Operating System. The system time can be changed using:

\vspace{-15pt}
\begin{minted}{console}
# timedatectl set-time
\end{minted}
\vspace{-10pt}

\noindent
The system can even be configured to use an NTP server to set time, using:

\vspace{-15pt}
\begin{minted}{console}
# timedatectl set-ntp yes
\end{minted}
\vspace{-10pt}

\noindent
This will enable the \textbf{chronyd} service. The chronyd service stores its config file in \verb|/etc/chrony.conf|, where we can set which NTP service we want to use. 

However, the system time isn't automatically written to the hardware clock, and thus, if the hardware clock's time is significantly different from the software time, after a reboot the changes made using \verb|timedatectl| will be lost. To write the system time back to the hardware clock, we use:

\vspace{-15pt}
\begin{minted}{console}
# hwclock --systohc
\end{minted}
\vspace{-10pt}	

\section{Setting up a Chrony Time Server}
To configure the time on RHEL 7, we need to configure chrony, using the \verb|timedatectl| command, recently introduced in this version. The current system time details can be obtained by:

\vspace{-15pt}
\begin{minted}{console}
# timedatectl status
Local time: Fri 2017-12-22 17:27:31 IST
Universal time: Fri 2017-12-22 11:57:31 UTC
RTC time: Fri 2017-12-22 11:57:31
Time zone: Asia/Kolkata (IST, +0530)
NTP enabled: yes
NTP synchronized: yes
RTC in local TZ: no
DST active: n/a
\end{minted}
\vspace{-10pt}

\noindent
If the current time-zone needs to be changed, we can use the command:

\vspace{-15pt}
\begin{minted}{console}
# timedatectl list-timezones
Africa/Abidjan
Africa/Accra
...
Pacific/Wallis
UTC
\end{minted}
\vspace{-10pt}

\noindent
After a new time-zone has been set, it can be verified with \verb|timedatectl status|. A new time-zone can be set using:

\vspace{-15pt}
\begin{minted}{console}
# timedatectl set-timezone Europe/Amsterdam 
# timedatectl status
Local time: Fri 2017-12-22 13:06:02 CET
Universal time: Fri 2017-12-22 12:06:02 UTC
RTC time: Fri 2017-12-22 12:06:02
Time zone: Europe/Amsterdam (CET, +0100)
NTP enabled: yes
NTP synchronized: yes
RTC in local TZ: no
DST active: no
Last DST change: DST ended at
Sun 2017-10-29 02:59:59 CEST
Sun 2017-10-29 02:00:00 CET
Next DST change: DST begins (the clock jumps one hour forward) at
Sun 2018-03-25 01:59:59 CET
Sun 2018-03-25 03:00:00 CEST
\end{minted}
\vspace{-10pt}

\subsection{NTP \& Chronyd Service}
The system can be configured to use a NTP server by setting the option \verb|timedatectl set-ntp yes|. This will start the chronyd service, if it wasn't already started. The current status of the chronyd service can be obtained with:

\vspace{-15pt}
\begin{minted}{console}
# systemctl status chronyd
● chronyd.service - NTP client/server
Loaded: loaded (/usr/lib/systemd/system/chronyd.service; enabled; vendor preset: enabled)
Active: active (running) since Fri 2017-12-22 10:29:48 IST; 7h ago
Docs: man:chronyd(8)
man:chrony.conf(5)
Main PID: 774 (chronyd)
CGroup: /system.slice/chronyd.service
└─774 /usr/sbin/chronyd

Dec 22 13:06:47 vmPrime.somuVMnet.com chronyd[774]: Source 139.59.43.68 online
Dec 22 13:06:47 vmPrime.somuVMnet.com chronyd[774]: Source 13.126.27.131 online
Dec 22 13:06:47 vmPrime.somuVMnet.com chronyd[774]: Source 139.59.21.22 online
\end{minted}
\vspace{-10pt}

\noindent
The location of the chronyd configuration files can be found by first finding out which package it comes from, and then performing a series of rpm queries. The \verb|rpm -qf <file>| command tells us which package contains that service (i.e., which package installed that file) while \verb|rpm -qc <package>| shows us the location of all the configuration files for that package. The method to determine the location of chrony's config files is:

\vspace{-15pt}
\begin{minted}{console}
# systemctl status chronyd
● chronyd.service - NTP client/server
Loaded: loaded (/usr/lib/systemd/system/chronyd.service; enabled; vendor preset: enabled)
Active: active (running) since Fri 2017-12-22 10:29:48 IST; 7h ago
Docs: man:chronyd(8)
man:chrony.conf(5)
Main PID: 774 (chronyd)
CGroup: /system.slice/chronyd.service
└─774 /usr/sbin/chronyd

Dec 22 13:06:47 vmPrime.somuVMnet.com chronyd[774]: Source 139.59.43.68 online
Dec 22 13:06:47 vmPrime.somuVMnet.com chronyd[774]: Source 13.126.27.131 online
...
# rpm -qf /usr/lib/systemd/system/chronyd.service
chrony-3.1-2.el7.centos.x86_64
# rpm -qc chrony-3.1-2.el7.centos.x86_64
/etc/chrony.conf
/etc/chrony.keys
/etc/logrotate.d/chrony
/etc/sysconfig/chronyd
\end{minted}
\vspace{-10pt}

\noindent
Thus the chronyd service can be managed by editing the config file \verb|/etc/chrony.conf|.