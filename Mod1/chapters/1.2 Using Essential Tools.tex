\chapter{Using Essential Tools}
\section{Man Command}

\textbf{man} followed by \textit{keyword} yields the manual page of that command. 
\vspace{-20pt}
\begin{minted}{console}
$ man ls
\end{minted}

\vspace{-15pt}
\noindent
\textbf{man} followed by option \textbf{-k} (for keyword) and then followed by a \textit{keyword} yields a list of all the commands containing that keywork and a brief descripiton of that command.
\vspace{-20pt} 
\begin{minted}{console}
$ man -k day
daylight (3)         - initialize time conversion information
dysize (3)           - get number of days for a given year
daylight (3p)        - set timezone conversion information
gettimeofday (2)     - get / set time
gettimeofday (3p)    - get the date and time
motd (5)             - message of the day
Net::Time (3pm)      - time and daytime network client interface
settimeofday (2)     - get / set time
Time::HiRes (3pm)    - High resolution alarm, sleep, gettimeofday, interval timers
\end{minted}

\vspace{-15pt}
\noindent
The numbers next to the commands indicate which seciton of the man pages the command belongs to (based on their functionality). The actual section that the commands belong to can be determined by the use of 

\vspace{-20pt}
\begin{minted}{console}
$ man man-pages
\end{minted}

\vspace{-15pt}
\noindent
The relevant sections for SysAdmins are Section 1, 5 \& 8. The sections are:

\vspace{-5pt}
\noindent
\begin{tabular}{p{0.1\textwidth}p{0.25\textwidth}p{0.6\textwidth}}
	\toprule
	Section Number & Deals with & Description \\
	\midrule
	1 & Commands (Programs) & Those  commands  that  can  be executed  by the user from within a	shell. \\
	2 & System calls & Those functions which must be performed by the kernel. \\		
	3 & Library calls & Most of the libc functions. \\		
	4 & Special files (devices) & Files found in /dev. \\		
	5 & File formats and conventions & The format for /etc/passwd and other human-readable files. \\		
	6 & Games & \\		
	7 & Overview, conventions, and miscellaneous & Overviews of various topics, conventions and  protocols,  character set standards, and miscellaneous other things. \\		
	8 & System management commands & Commands like mount(8), many of which only root can execute.	\\
	\bottomrule
\end{tabular}

\noindent
To filter down the output of the \textbf{man -k} command, we can use \textbf{grep} to obtain only the relevant parts of the result on the basis of the appropriate section number in the man-pages. 

\noindent
This can be achieved using the pipe which feeds the output of the first command to the input of the second command. 

\vspace{-20pt}
\begin{minted}{console}
$ man -k day | grep 3
daylight (3)         - initialize time conversion information
dysize (3)           - get number of days for a given year
daylight (3p)        - set timezone conversion information
gettimeofday (3p)    - get the date and time
Net::Time (3pm)      - time and daytime network client interface
Time::HiRes (3pm)    - High resolution alarm, sleep, gettimeofday, interval timers
\end{minted}

\vspace{-20pt}
\noindent

\section{Understanding Globbing and Wildcards}
\begin{tabular}{rcl}
	\textbf{*} &- &Indicates any string. \\
	\textbf{?} &- &Indicates any single character. \\
	\textbf{[$\cdots$]} &- &Indicates any character provided within brackets. \\
	\textbf{[!$\cdots$]} &- &Indicates any character \textit{NOT} provided within brackets. \\
	\textbf{[a-f]} &- &Indicates any character provided within the range of a to f. \\
\end{tabular}

\section{Understanding Globbing and Wildcards}
\noindent
\begin{tabular}{lcp{0.8\textwidth}}
	\textbf{\$ ls a*} &- &Lists all files and folders (including contents of each folder) that start with "a" \\
	\textbf{\$ ls *a*} &- &Lists all files and folders that contain the string "a". \\
	\textbf{\$ ls -d a*} &- &Shows all files and folders that start with "a" but excludes the contents of each individual folder. \\
	\textbf{\$ ls ??st*} &- &Lists all files and folders that have "st" as the $3^{rd}$ and the $4^{th}$ character in their name. \\
	\textbf{\$ [a-f]} &- &Indicates any character provided within the range of a to f. \\
\end{tabular}

\section{Understanding I/O Redirection and Pipes}
\subsection{I/O Redirection}
File Descriptors: 

\noindent
\begin{tabular}{rccclcp{0.5\textwidth}}
	\textbf{STDIN} &- &0 &- &Standard Input &- &Represents the "file" for the Standard Input Device (generally Keyboard). \\
	\textbf{STDOUT} &- &1 &- &Standard Output &- &Represents the "file" for the Standard Output Device (generally the Monitor). \\
	\textbf{STDERR} &- &2 &- &Standard Error &- &Represents the "file" for the Standard Output Device (also, generally the Monitor). \\
\end{tabular}

\noindent
Redirection:

\noindent
\begin{tabular}{rclcp{0.7\textwidth}}
	\textbf{STDIN} &- &$<$ &- &Feeds the file to the right of the "<" as input to the command on the left. \\
	\textbf{STDOUT} &- &$>$ &- &Stores the output of the command to the left of the ">" to the file indicated on the right. \textit{OVERWRITES} the mentioned file.\\
	\textbf{STDOUT} &- &$>>$ &- &Stores the output of the command to the left of the "$>>$" to the file indicated on the right. \textit{APPENDS} the mentioned file.\\
	\textbf{STDERR} &- &$2>$ &- &Redirects the errors from the command mentioned on the left to the file on the right. \textit{OVERWRITES} the mentioned file.\\
\end{tabular}
	
\noindent
\begin{minted}{console}	
$ mail -s hi root < .
$ ls > myFile
$ ls -lh >> myFile
$ grep hi * 2> /dev/tty6
\end{minted}

\noindent
1 - \textit{mail} is a simple command used to send messages. The command expects the message to terminate with a ".", so we feed it directly to the command, instead of providing any input.

\noindent
4 - The STDERR is redirected to tty6 (a virtual terminal connected to the host). Can also be diverted to a file if needed, such as an errorLog.

\subsection{Piping}
The command \verb|$ ps aux| shows us the overview of all the running processes on the host. However, it's too long to view all at once. In such situations, or wherever we need to feed the output of the first command to the input of the second command, we use the pipe operator. The command would then be \verb=$ ps aux | less=.

The difference in the usage of the piping and redirection operators is that Pipe is used to pass output to another program or utility, while Redirect is used to pass output to either a file or stream.

\section{Using I/O Redireciton and Piping}

\noindent
\begin{minted}{console}	
$ ps aux | awk '{print $2}'
$ ps aux | awk '{print $2}' | sort
$ ps aux | awk '{print $2}' | sort -n
\end{minted}

\noindent
The second column (\$2) of the \verb|$ ps aux| command contains the Process ID (PID), and if we only want to filter the output such that only the PID is shown, we simply use the \textbf{awk} filtering utility. 

If we want to sort the output of the command, we use the \textbf{sort} utility, but it generally sorts as a string. To sort the output as a number, we use the option \textbf{sort -n}.

\noindent
If you expect lots of errors for a particular command, but want to discard all errors and only see the output when successful, then simply redirect the STDERR to \verb|/dev/null|, which is a special device that discards all data written to it, i.e., a dustbin for data. 

\begin{minted}{console}	
$ find / -name "*.rpm"
$ find / -name "*.rpm" 2> /dev/null
\end{minted}

\noindent
The first command shows all output including errors, but the second command discards all errors and shows the rest. 	

\begin{minted}{console}
$ some_command > /dev/null 2> &1
\end{minted}

\noindent
The above code redirects STDOUT to \verb|/dev/null| thus destroying the output, and also redirects the STDERR (2>) to STDOUT (\&1). Essentially, it discards all output - useful when we don't need the output but only need the command to execute.

\begin{minted}{console}
$ ls / > file_list.txt
$ sort < file_list.txt > file_list_sorted.txt
\end{minted}

\noindent
The above command stores the contents of the root directory in \textit{file\_list.txt}. Then, the second command uses both input and output redirection! The input of the sort command is fed from \textit{file\_list.txt} and the corresponding output sent to \textit{file\_list\_sorted.txt}.