\chapter{Working with Text Files}
\section{Understanding Regular Expressions}
\begin{figure}[h]
	\centering
	\includegraphics[width=\linewidth]{{{D:/SysAdmin & Dev/SysAdmin/RedHat/RHCSA Complete Video Course/1. Installing RHEL Server/1.4. Working with Text Files/1.4.1 (a) RegEx Cheat Sheet}}.jpg}
	\caption{RegEx Cheat Sheet}
	\label{fig:regex-cheat-sheet}
\end{figure}

\section{Using common text tools}
\subsection{cat}
The \verb|cat| command prints the entire content of a file on to the terminal. 

\subsection{less}
Sometimes the \verb|cat| command is unsuitable, like in the case of extremely large files. In such cases, like the \verb|/var/log/messages|, the default system log file, using \verb|cat| won't work as the majority of the messages would scroll past fast. For such cases, \verb|less| is a better utility. Search functionality is exactly the same as in the case of vim. 

\subsection{Head and Tail}
\subsubsection{Head}
The \verb|head| command by default shows us the first 10 lines of a text file. To see more or less lines, the \verb|-n| option can be used.

\begin{minted}{console}
$ head -n 20 file.txt
\end{minted}

\subsubsection{Tail}
The \verb|tail| command by default shows us the last 10 lines of a text file. To see more or less lines, the \verb|-n| option can be used.

\begin{minted}{console}
$ tail -n 5 file.txt
\end{minted}

\subsubsection{Combination of head and tail}
The combination of these two commands can enable the viewing of text in between specific line numbers. The command below shows lines 16-20 of \texttt{file.txt}

\begin{minted}{console}
$ head -n 20 file.txt | tail -n 5
\end{minted}

\subsection{cut}
With the \verb|cut| utility, we can print out a specific column form a text file. It assumes the columns are separated by Tabs. Which specific column is to be printed is set by using the \verb|-f| option. For example, to only print the first column of a text file, we say:

\begin{minted}{console}
$ cut -f 1 cities
\end{minted}

\noindent
To provide a different delimter, such as ":" we use the \verb|-d| option followed by the delimiter of our choice. 

\begin{minted}{console}
$ cut -f 1 -d : /etc/passwd
\end{minted}

\subsection{sort}
This command sorts the input provided in the order of the ASCII table. That means numbers first, captial letters next and finally the lower case letters. 

\begin{minted}{console}
$ cut -f 1 -d : /etc/passwd | sort
\end{minted}

\noindent
To sort on the basis of a specific criteria:

\begin{tabular}{lcp{0.85\textwidth}}
	\textbf{-n} &- &Sort on the basis of actual numberical value, instead of treating a number as a string. \\
	\textbf{-f} &- &Sort in a case insensitive manner. \\
\end{tabular}

\subsection{tr}
The \verb|tr| command replaces certain characters with certain other characters. Thus, it's frequently used in conjunction with pipes to modify the output of a command. 

\begin{minted}{console}
$ echo hello | tr a-z A-Z
HELLO
$ echo hello | tr [:lower:] [:upper:]
HELLO
\end{minted}

\section{grep}
\verb|grep| is a filtering utility that only prints those lines that contain a certain expression matching the pattern provided by a \textit{RegEx}.

\begin{minted}{console}
$ ps aux | grep tracker
somu      10450  0.0  0.4 469796  9000 ?        SNl  10:06   0:00 /usr/libexec/tracker-miner-user-guides
somu      10465  0.0  0.6 536856 12012 ?        Sl   10:06   0:00 /usr/libexec/tracker-store
somu      10611  0.0  0.7 779816 13108 ?        SNl  10:06   0:00 /usr/libexec/tracker-extract
somu      10614  0.0  0.5 469800  9632 ?        SNl  10:06   0:00 /usr/libexec/tracker-miner-apps
somu      10615  0.0  0.7 710160 13204 ?        SNl  10:06   0:00 /usr/libexec/tracker-miner-fs
root      17396  0.0  0.0 112644   968 pts/0    R+   13:49   0:00 grep --color=auto tracker
\end{minted}

\noindent
Another use for \verb|grep| is searching files. The syntax is \verb|grep <filename-pattern> <search-directory>|.

To avoid errors notifying "is a directory", simply redirect errors to \verb|/dev/null|.

\begin{minted}{console}
$ grep lisa * 2> /dev/null
group:lisa:x:1001:
gshadow:lisa:!::
passwd:lisa:x:1001:1001::/home/lisa:/bin/bash
passwd-:lisa:x:1001:1001::/home/lisa:/bin/bash
services:na-localise     5062/tcp                # Localisation access
services:na-localise     5062/udp                # Localisation access
shadow:lisa:$6$0l/zSJkh$xjJNYNnj1rPs7Fq0hDWt8VucS0nLL82XrMYpmBnLF2DrzB2npFvCwxM9MJEHgCHCwvabCgEA17LK2aU0h9FIT/:17414:0:99999:7:::
shadow-:lisa:password:17414:0:99999:7:::
\end{minted}

\subsection{wc}
Counts the number of words, lines and characters. 

\begin{tabular}{lcl}
	\textbf{-l} &- &Counts the number of lines \\
	\textbf{-w} &- &Counts the number of words \\
	\textbf{-m} &- &Counts the number of characters \\
	\textbf{-c} &- &Counts the number of bytes \\
	\textbf{-L} &- &Counts the length of the longest line \\
\end{tabular}

To see the number of matched lines using \verb|grep|, simply use:

\begin{minted}{console}
$ ps aux | wc
188    2344   19581
\end{minted}

\subsection{grep -l}
Grep by default returns the name of the matching file followed by the matching lines. This output can be made more readable by \verb|grep -l| which lists all the files in the directory that matches the criteria.

\begin{minted}{console}
$ grep lisa * 2> /dev/null
group:lisa:x:1001:
passwd:lisa:x:1001:1001::/home/lisa:/bin/bash
services:na-localise     5062/tcp                # Localisation access
services:na-localise     5062/udp                # Localisation access
$ grep -l lisa * 2> /dev/null
group
passwd
services
\end{minted}

\subsection{grep -i}
The \textbf{-i} flag turns the grep command case-insensitive!

\begin{minted}{console}
$ grep lisa * 2> /dev/null
group:lisa:x:1001:
passwd:lisa:x:1001:1001::/home/lisa:/bin/bash
services:na-localise     5062/tcp                # Localisation access
services:na-localise     5062/udp                # Localisation access
$ grep -i lisa * 2> /dev/null
group:lisa:x:1001:
passwd:lisa:x:1001:1001::/home/lisa:/bin/bash
services:ltctcp          3487/tcp                # LISA TCP Transfer Channel
services:ltcudp          3487/udp                # LISA UDP Transfer Channel
services:na-localise     5062/tcp                # Localisation access
services:na-localise     5062/udp                # Localisation access
\end{minted}

\subsection{grep -R}
The usage of the \verb|-R| flag puts grep in recursive mode, where the utility searches for the file in each subfolder as well. 

\begin{minted}{console}
$ grep -iR lisa * 2> /dev/null
group:lisa:x:1001:
lvm/lvm.conf:	# If using external locking (type 2) and initialisation fails, with
passwd:lisa:x:1001:1001::/home/lisa:/bin/bash
Binary file pki/ca-trust/extracted/java/cacerts matches
Binary file pki/java/cacerts matches
...
\end{minted}

\subsection{grep -v}
Grep with a \verb|-v| flag excludes the matching results. Here we can exclude the lines containing "Binary" using:

\begin{minted}{console}
$ grep -iR lisa * 2> /dev/null | grep -v Binary
alternatives/jre_openjdk/lib/security/nss.cfg:handleStartupErrors = ignoreMultipleInitialisation
alternatives/jre_openjdk_exports/lib/security/nss.cfg:handleStartupErrors = ignoreMultipleInitialisation
brltty/fr-abrege.ctb:word civilisation	14-1236-16
brltty/fr-abrege.ctb:word civilisations	14-1236-16-234
brltty/fr-abrege.ctb:word généralisation	1245-1345-16
brltty/latex-access.ctb:  brailleTranslator.capitalisation = "6dot"
brltty/latex-access.ctb:      
passwd:lisa:x:1001:1001::/home/lisa:/bin/bash
sane.d/canon_pp.conf:# Set a default initialisation mode for each port.  Valid modes are:
services:ltctcp          3487/tcp                # LISA TCP Transfer Channel
services:ltcudp          3487/udp                # LISA UDP Transfer Channel
...
\end{minted}

\section{sed and awk basics}
\subsection{sed}
\verb|sed| is an old utility that's used to process text. Many of it's functionalities can now be done using \verb|grep| itself. 

\subsubsection{sed q}
To see the first two lines of a file using \verb|sed| we use:

\begin{minted}{console}
$ sed 2q /etc/passwd
root:x:0:0:root:/root:/bin/bash
bin:x:1:1:bin:/bin:/sbin/nologin
\end{minted}

\subsubsection{sed -n}
The \verb|-n| flag makes \verb|sed| print no output unless the \verb|p| flag is also provided. Here, we use a Regular Expression \textit{"\^root"} to match only a certain part of the text and then the \verb|p| flag to print only if that criteria is matched. 

\begin{minted}{console}
$ sed -n /^root/p /etc/passwd
root:x:0:0:root:/root:/bin/bash
\end{minted}

The above result could also be obtained with \verb|grep "^root" /etc/passwd|.

\subsubsection{Substitution with sed}
Sed can be used to substitute text within a file using the \verb|s| parameter. It's used as :

\begin{minted}{console}
$ cat names
Somu
Arpi
Neha
Santy
Debu
$ sed -i 's/Santy/Dickwad/g' names
$ cat names
Somu
Arpi
Neha
Dickwad
Debu
\end{minted}

The \verb|-i| flag asks the modifications to be made in place. Otherwise the output (changed text) would've simply been displayed to the screen and then need to be redirected to a file for storage.

\subsection{awk}
awk is another utility that is especially useful when working with text files. It excels at operations like cutting out information.

\subsubsection{Cutting out information using awk}
For certain operations, the \verb|awk| command is a lot more powerful than the \verb|cut| utility. In the example below \verb|cut| has a hard time recognizing the second field, while \verb|awk| has no problem whatsoever!

\begin{minted}{console}
$ ps aux | grep 'gdm'
root       1117  0.0  0.1 480248  4688 ?        Ssl  09:20   0:00 /usr/sbin/gdm
root       1333  0.8  1.2 328524 47220 tty1     Ssl+ 09:20   0:43 /usr/bin/X :0 -background none -noreset -audit 4 -verbose -auth /run/gdm/auth-for-gdm-bgrBZH/database -seat seat0 -nolisten tcp vt1
root       1718  0.0  0.1 528944  5848 ?        Sl   09:20   0:00 gdm-session-worker [pam/gdm-password]
gdm        1737  0.0  0.1 458088  4152 ?        Sl   09:20   0:00 ibus-daemon --xim --panel disable
gdm        1741  0.0  0.1 373560  5424 ?        Sl   09:20   0:00 /usr/libexec/ibus-dconf
gdm        1745  0.0  0.2 438152  7772 ?        Sl   09:20   0:00 /usr/libexec/ibus-x11 --kill-daemon
somu      12218  0.0  0.0 112664   972 pts/0    R+   10:47   0:00 grep --color=auto gdm
$ ps aux | grep 'gdm' | cut -f 2
root       1117  0.0  0.1 480248  4688 ?        Ssl  09:20   0:00 /usr/sbin/gdm
root       1333  0.8  1.2 328524 47220 tty1     Ssl+ 09:20   0:43 /usr/bin/X :0 -background none -noreset -audit 4 -verbose -auth /run/gdm/auth-for-gdm-bgrBZH/database -seat seat0 -nolisten tcp vt1
root       1718  0.0  0.1 528944  5848 ?        Sl   09:20   0:00 gdm-session-worker [pam/gdm-password]
gdm        1737  0.0  0.1 458088  4152 ?        Sl   09:20   0:00 ibus-daemon --xim --panel disable
gdm        1741  0.0  0.1 373560  5424 ?        Sl   09:20   0:00 /usr/libexec/ibus-dconf
gdm        1745  0.0  0.2 438152  7772 ?        Sl   09:20   0:00 /usr/libexec/ibus-x11 --kill-daemon
somu      12226  0.0  0.0 112664   968 pts/0    R+   10:48   0:00 grep --color=auto gdm
$ ps aux | grep 'gdm' | awk '{ print $2 }'
1117
1333
1718
1737
1741
1745
12248
\end{minted}