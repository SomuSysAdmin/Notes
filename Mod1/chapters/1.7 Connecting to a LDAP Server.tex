\chapter{Connecting to a LDAP Server}
\section{Understanding LDAP}
LDAP is an easy way to provide centralized authentication from a server. This way, many computers can be connected to a single LDAP server and the user accounts (and permissions) have to be set up only once!

\textbf{LDAP} stands for \textit{Lightweight Directory Access Protocol}. It connects us to a hierarchical directory server. In the hierarchy (e.g., server.rhatcertification.com), there are top level domains such as \textit{.com}, subdomain (rhatcertification) and leaf objects (lisa). Even though the structure is similar to DNS, the notation of LDAP is different. For every container object, we write \verb|dc=<objectName>| (\textit{dc} $\rightarrow$ Domain Component) and for leaf objects, it becomes \verb|cn=<objectName>| (\textit{cn} $\rightarrow$ Common Name). The complete format then becomes \textit{cn=lisa,dc=rhatcertificaton,dc=com}.

An important part of connecting to an LDAP server is the \textbf{base context}. The base context, like the search domain of DNS, is the starting point where our client should look for objects. In this case, the base context is \verb|dc=rhatcertification,dc=com|. Thus, for logging in to a server, the \textit{cn}(lisa) is searched for within the base context. 

\subsection{/bin/login}
The \verb|login| service is used whenever the user requires authentication to connect to anything. 

\subsection{ldd}
The \verb|ldd| command (List Dynamic Dependencies) prints all the shared libraries required by a program. 

\begin{minted}{console}
$ ldd /bin/login
linux-vdso.so.1 =>  (0x00007ffc333e3000)
libpam.so.0 => /lib64/libpam.so.0 (0x00007f85cad8a000)
libpam_misc.so.0 => /lib64/libpam_misc.so.0 (0x00007f85cab86000)
libaudit.so.1 => /lib64/libaudit.so.1 (0x00007f85ca95d000)
libselinux.so.1 => /lib64/libselinux.so.1 (0x00007f85ca736000)
libc.so.6 => /lib64/libc.so.6 (0x00007f85ca373000)
libdl.so.2 => /lib64/libdl.so.2 (0x00007f85ca16e000)
libcap-ng.so.0 => /lib64/libcap-ng.so.0 (0x00007f85c9f68000)
libpcre.so.1 => /lib64/libpcre.so.1 (0x00007f85c9d06000)
/lib64/ld-linux-x86-64.so.2 (0x0000558e5f0bd000)
libpthread.so.0 => /lib64/libpthread.so.0 (0x00007f85c9ae9000)
\end{minted}

\noindent
The PAM library shown above (libpam) is akin to a plugin that adds additional functionality to the \verb|login| utility (as well as several others). PAM stands for \textit{Plugable Authentication Modules}. The configuration files for the authentication module is stored in \verb|/etc/pam.d| directory. The \verb|/etc/pam.d/login| is the configuration file for \verb|login| utility. 

\subsection{PAM config file syntax}
The PAM config files are each named after the services that require the usage of PAM. For example, the config file for the \textit{login} service is called \verb|/etc/pam.d/login|. Each file lists a bunch of rules in the syntax : \verb|<service-type> <control> <module-path> <arugments>|.

\subsubsection{Service Type}
\vspace{-10pt}
\begin{tabular}{ll}
	\toprule
	\textbf{Service Type} &\textbf{Description} \\
	\midrule
	\textbf{auth} &Deals with user authentication via password (or other means like keys). \\	
	\textbf{account} &Non-authentication based account management. \\	
	\textbf{password} &Updating the authentication token of the user. \\
	\textbf{session} &Modules listed here are used for setup/cleanup of a service for the user. \\
	\bottomrule
\end{tabular}

\subsubsection{PAM Module Controls}
\vspace{-10pt}
\begin{tabular}{ll}
	\toprule
	\textbf{Control} &\textbf{Description} \\
	\midrule
	\textbf{requisite} &Immediately causes failure when the module returns a status that isn't 'success'. \\	
	\textbf{required} &If the service returns a non-success status, then the operation fails ultimately, \\&but only after the modules below it are invoked. This is to prevent a person with \\&malicious intent from gaining knowledge of which module failed. \\	
	\textbf{sufficient} &If a \textit{sufficient} module returns a 'success' status, the other modules below it that \\&are also a part of 'sufficient' management group will not be invoked. In case of \\&failure, another module listed 'sufficient' in the stack below it must succeed for \\&the operation to succeed. \\
	\textbf{optional} &Only causes failure if the rule stack contains only optional modules and all fail. \\
	\midrule
	\textbf{include} &For the given service type, include all lines of that type from the provided \\&configuration file. \\
	\textbf{substack} &Same as \textit{include} but when \textit{done} and \textit{die} actions are evaluated, they only cause \\&skipping of the substack. \\
	\bottomrule
\end{tabular}

The \verb|login| config file in \verb|/etc/pam.d| contains the line:
\vspace{-10pt}
\begin{minted}{bash}
auth       substack     system-auth
\end{minted}

\noindent
\textbf{NOTE} \textit{: the entry for pam\_ldap requires that the host should be able to use LDAP, which requires} \verb|pam_ldap| \textit{to be installed, and} \verb|authconfig-tui| \textit{to be executed.}

The \verb|system-auth| file has rules for the common login procedure for any any process that deals with user authentication. This file in turn contains the lines :

\begin{minted}{bash}
auth        sufficient    pam_unix.so nullok try_first_pass
...
auth        sufficient    pam_ldap.so use_first_pass
\end{minted}

The line \verb|auth  sufficient  pam_unix.so| tells the system to look at the System login local authentication mechanism (\verb|pam_unix.so|). If that is not successful, the system is instructed to use the LDAP PAM mechanism (in \verb|pam_ldap.so|). Thus, the login process is contacting an LDAP server and trying to verify if the user account exists on that server. 

\begin{figure}[H]
	\centering
	\includegraphics[width=0.9\linewidth]{"chapters/1.6.a LDAP Authentication"}
	\caption{LDAP Authentication}
	\label{fig:1}
\end{figure}

\noindent
Next, the LDAP configuration file (\verb|/etc/nslcd.conf|) is read, which contains the URI of the LDAP Server (\verb|ldap://server.rhatcert.com|). Finally, the client is able to connect to the LDAP server where there is a LDAP hierarchy that it can log into.  

\section{Telling your server where to find the LDAP Server}

\subsection{nscd}
The Naming Service Cache Daemon needs to be installed to configure the connection of a server to an external LDAP server. It is the part of the OS that caches the information from external authentication mechanisms on the local machine.

\subsection{nss-pam-ldapd}
This sets up the local name resolution and local authentication and connects it to LDAP services. 

\subsection{pam\_ldap}
The libraries needed to make the local authentication aware of LDAP services. 

\subsection{authconfig-gtk}
\verb|authconfig| is an utility used to setup the server for external authentication. There are several variations of it, such as \verb|authconfig|, \verb|authconfig-tui| and \verb|authconfig-gtk| (GUI based).

\subsubsection{LDAP Search Base DN}
The search base DN consists of Domain Components (dc) with commas as separators. Example : \verb|dc=rhatcertification.com,dc=com|. 

\subsubsection{LDAP Server}
The server needs to have a matching certificate to the one that the client receives on connection. This is only possible with a domain name, and not an IP address. The reason for this is the server name has to match the one in the certificate and the certificate can only have one name associated to it. 

\subsubsection{TLS Certificate}
The use of a Transport Layer Security (TLS) certificate is important because unless it's used, the LDAP password is sent across the network unencrypted, which makes the entire system vulnerable. We also need to download the TLS certificate from the server (e.g., \verb|ftp://server.rhatcertification.com/pub/slapd.pem|).

\subsection{Switching to an LDAP user}
The user can switch to an LDAP user just as easily as a local user using:

\begin{minted}{console}
$ su - <ldap_username>
\end{minted}

\section{Understanding Automount}
While it's possible to have the LDAP users use local directories on the server, generally an NFS hosts the home directories of these users. Thus, we have to automount the home directories of these users as if they're part of the local file system. 

Let us consider a system where automounting is enabled, and the user wants to access a folder \verb|/data/files|. If the folder \verb|/data| is hosted on a remote file system and monitored by the automount process (called \textbf{autofs}), then there will have a file called \verb|/etc/auto.master| containing the line:

\begin{minted}{bash}
/data	/etc/auto.data
\end{minted}

The \verb|/etc/auto.master| file only shows that the automount process recognizes the \verb|/data| directory as an automount directory. This merely states that the mounting details for the data folder is present in its own file called \verb|/etc/auto.data|. That file will contain:

\begin{minted}{bash}
files	-rw		nfsServer:/data
\end{minted}

The files directory is a subdirectory of the \verb|/data| directory, and thus when the \verb|/files| directory needs to be accessed, an NFS mounting operation needs to occur, with read write access on the nfsServer's (hostname) \verb|/data| directory. Even though the user will be working on the NFS server, he/she will have no inkling of this happening behind the scene. 

\begin{figure}[H]
	\centering
	\includegraphics[width=0.9\linewidth]{"chapters/1.7.a NFS Automount"}
	\caption{NFS Automount}
	\label{fig:1}
\end{figure}


\subsection{Server selection for auto-mounting}
Primarily two types of servers can accomplish the auto-mounting of home directories for LDAP users - NFS and Samba servers. In case of NFS server, the files will only be available on the local network. For access through the Internet, a Samba server has to be used. 

\subsection{Samba server's CIFS protocol to automount}
Let us consider an LDAP user \textit{ldapuser1} who has his home directory configured to \\\verb|/home/guests/ldapuser1| in his user properties. When the user logs in, there will be a system call to go to the home directory for the users, which in turn calls \verb|autofs| to mount the file system. It'll consult the \verb|/etc/auto.master| file to find:

\begin{minted}{bash}
/home/guests	/etc/auto.guests
\end{minted}

\noindent
If anyone wants to visit that directory, the process should consult the \verb|/etc/auto.guests| file, containing the mounting details with the UNC (Universal Naming Convention) path of the actual Samba server on the internet. 

\begin{minted}{bash}
*	-fstype=cifs,username=ldapusers,password=password\ 
		://server.rhatcertification.com/data/&
\end{minted}

\noindent
So, if anyone goes to \verb|*| (i.e., any directory in \verb|/data/guests|), like \verb|/home/guests/ldapuser1| or \verb|/home/guests/ldapuser2| and so on, a CIFS (Common Internet File Sharing protocol, which uses Server Message Block [SMB, Used by Samba]) mount needs to occur, specified by \verb|fstype=cifs|, with the given username and password. The address of the server is then provided. 

What is of particular importance here is the matching of \verb|*| and \verb|&|. While the \verb|*| wildcard selects whatever folder the user tried to enter, the \verb|&| in the address is replaced with the corresponding text from user. Thus, if the user visits \verb|/home/guests/ldapuser1|, the \verb|*| is replaced with \verb|ldapuser1| and a matching folder is searched for on the server. 

\begin{figure}[H]
	\centering
	\includegraphics[width=0.9\linewidth]{"chapters/1.7.b Samba Automount"}
	\caption{Samba Automount}
	\label{fig:1}
\end{figure}


\section{Configuring Automount}
To use automount, the \textit{automount service} in the \textbf{autofs} package needs to be installed. The primary configuration file is \verb|/etc/auto.master|. To automount the \verb|/etc/guests| folder from a Samba server, we just need to specify in the file that:

\begin{minted}{bash}
/home/guests	/etc/auto.guests
\end{minted}

Now we can add the mount options in its own individual file \textit{auto.guests}, such that quick mounting and unmounting is possible. 

\subsubsection{Configuration on the Samba Server}
The Samba server containing the \verb|data| directory needs to have the following configuration in its \verb|/etc/samba/smb.conf| :

\vspace{-10pt}
\begin{minted}{bash}
[data]
comment = LDAP Users' home directories
path = /home/guests
public = yes
writable = no
\end{minted}

\subsection{NFS Server Automounting}

In the case of a NFS mounted directory, the \textit{auto.guests} file would look like:

\begin{minted}{bash}
*	-rw	nfsServer.domain.com:/home/guests/&
\end{minted}

\noindent
In case of either servers, the syntax remains the same. First we provide the name of the directory (\verb|*|), then the mounting options (e.g., \verb|-rw| in case of NFS) and finally the path to the real directory that has to be mounted on the local file system from that server.

\section{Configuring NFS and Automount}
\subsection{yum search}
\vspace{-10pt}
The \verb|yum| utility provides a searching function that searches the name, description, summary and url of all the packages available for a keyword. 

\vspace{-15pt}
\begin{minted}{console}
# yum search nfs
Loaded plugins: fastestmirror, langpacks
Loading mirror speeds from cached hostfile
* base: mirror.digistar.vn
* extras: mirror.dhakacom.com
* updates: mirror.digistar.vn
=================================== N/S matched: nfs ====================================
libnfsidmap.i686 : NFSv4 User and Group ID Mapping Library
libnfsidmap.x86_64 : NFSv4 User and Group ID Mapping Library
libnfsidmap-devel.i686 : Development files for the libnfsidmap library
libnfsidmap-devel.x86_64 : Development files for the libnfsidmap library
nfs-utils.x86_64 : NFS utilities and supporting clients and daemons for the kernel NFS server
nfs4-acl-tools.x86_64 : The nfs4 ACL tools
nfsometer.noarch : NFS Performance Framework Tool
nfstest.noarch : NFS Testing Tool
\end{minted}

\subsection{Creating an NFS Server}
The \textbf{nfs-utils} package is needed to setup an NFS server. The NFS configuration file is called \verb|/etc/exports|. It specifies which file systems are exported to remote hosts (from the NFS server's perspective) and provides their respective mounting options. The contents of the \textit{exports} file has to follow the syntax : 

\begin{minted}{bash}
/data	*(rw,no_root_squash)	
\end{minted}
\vspace{-10pt}

\noindent
Here, \verb|/data| is the name of the directory to be hosted on the NFS, with read/write permissions from all (\verb|*|) IP addresses on the local network (since NFS only works on the local network). In cases it's not desirable to have the local machine's administrator act as the admin of the NFS server, then the way to perform this is called root squashing. In our case, we turn it off as we want the root user to retain administrative privileges on the NFS server as well. 

\subsection{Starting the NFS server}
To start the service corresponding to the NFS server, we use the \verb|systemctl| command. 

\begin{minted}{console}
$ systemctl start nfs
\end{minted}

\noindent
In case the service fails to start, the following command can provide hints about what went wrong :

\begin{minted}{console}
# systemctl status -l nfs
nfs-server.service - NFS server and services
Loaded: loaded (/usr/lib/systemd/system/nfs-server.service; disabled; vendor preset: disabled)
Active: active (exited) since Tue 2017-11-21 10:00:04 IST; 24s ago
Process: 3178 ExecStart=/usr/sbin/rpc.nfsd $RPCNFSDARGS (code=exited, status=0/SUCCESS)
Process: 3173 ExecStartPre=/bin/sh -c /bin/kill -HUP `cat /run/gssproxy.pid` (code=exited, status=0/SUCCESS)
Process: 3172 ExecStartPre=/usr/sbin/exportfs -r (code=exited, status=1/FAILURE)
Main PID: 3178 (code=exited, status=0/SUCCESS)
CGroup: /system.slice/nfs-server.service

Nov 21 10:00:04 ldapserver.somuvmnet.local systemd[1]: Starting NFS server and services...
Nov 21 10:00:04 ldapserver.somuvmnet.local exportfs[3172]: exportfs: Failed to stat /data: No such file or directory
Nov 21 10:00:04 ldapserver.somuvmnet.local systemd[1]: Started NFS server and services.
\end{minted}
\vspace{-10pt}

Now we can see the mounting status of the NFS server hosted at localhost, using the \verb|showmount -e localhost| command. Further, the NFS folder can be mounted manually using the \verb|mount| command. 

\vspace{-15pt}
\begin{minted}{console}
# showmount -e localhost
Export list for localhost:
/data *
# mount localhost:/data /mnt/nfs
# ls /nfs
localNFSfile1  localNFSfile2  localNFSfile3
\end{minted}

\subsection{Automounting NFS}
For automounting NFS, we create a new entry in \verb|auto.master| for a file \verb|/etc/auto.nfs| :

\vspace{-15pt}
\begin{minted}{bash}
/mnt/nfs	/etc/auto.nfs
\end{minted}
\vspace{-10pt}

\noindent
This file contains the following mounting details:	
\vspace{-15pt}
\begin{minted}{bash}
files	-rw	localhost:/data/
\end{minted}
\vspace{-10pt}

\noindent Thus, when the user enters the \textit{files} directory within the \textit{nfs} directory, he'll find the same files as in the \verb|/data| directory of localhost. 

\vspace{-15pt}
\begin{minted}{console}
# cd nfs
# ls
# ls -lha
total 0
drwxr-xr-x. 2 root root  0 Nov 21 11:59 .
drwxr-xr-x. 3 root root 17 Nov 21 11:59 ..
# cd files
# ls
nfs1  nfs2  test1
# cd ..
# ls
files
\end{minted}

\noindent	Note that the \verb|/nfs/files| directory isn't actually created before the user tries to enter the \textit{files} directory. 

\section{Modifying nslcd Configuration}
\subsection{Naming Services LDAP Client Daemon}
The \verb|nslcd| is a service that connects the local file system to the external LDAP server. The status of the \verb|nslcd| can be checked using:

\vspace{-15pt}
\begin{minted}{console}
$ systemctl status nslcd
nslcd.service - Naming services LDAP client daemon.
Loaded: loaded (/usr/lib/systemd/system/nslcd.service; enabled; vendor preset: disabled)
Active: active (running) since Mon 2017-11-20 12:03:06 IST; 5h 59min ago
Process: 1108 ExecStart=/usr/sbin/nslcd (code=exited, status=0/SUCCESS)
Main PID: 1151 (nslcd)
CGroup: /system.slice/nslcd.service
	1151 /usr/sbin/nslcd
\end{minted}
\vspace{-10pt}

\subsubsection{/etc/nsswitch.conf}
For every LDAP user, their identity needs to be known to the local system. This is based on the configuration stored in \verb|/etc/nsswitch.conf|. In this file, there is a line:

\vspace{-15pt}
\begin{minted}{bash}
passwd:	files	sss	ldap
\end{minted}
\vspace{-10pt}

This represents the order in which the sources of user account are searched for user related information. \verb|sss| is an older service no longer used by RHEL-7. Finally, it looks for the information in LDAP using \verb|nslcd|.

\subsubsection{PAM using nslcd}
PAM is responsible for the actual authentication system that ensures the LDAP server is known to the authentication mechanism. This entire process is achieved using \textit{nslcd}. 

\subsection{/etc/nslcd.conf}
This file contains the information stored by using the \verb|authconfig-gtk| command and has options to configure the \verb|nslcd| such that the LDAP server can be connected to and used. 
	
\noindent
\begin{tabular}{lll}
	\toprule
	\textbf{Option} &\textbf{Description} &\textbf{Example Value}\\
	\midrule
	uri &The Uniform Resource Identifier of the  &ldap://server.rhatcertification.com \\
	&LDAP Server. & \\
	base &The base context of the LDAP Server &dc=rhatcertification,dc=com \\
	ssl &Whether to use SSL/TLS. If start\_tls is &start\_tls \\
	&given, via the use of StartTLS, an & \\
	&insecure connection is upgraded to a &\\
	& secure one. &\\
	tls\_cacertdir &Location of the downloaded certificate &\verb|/etc/openldap/cacerts| \\
	&of the LDAP Server. &\\
	\bottomrule
\end{tabular}

In case of any problems with using LDAP, the \verb|/var/log/messages| file may contain hints that may indicate what's wrong. 