\chapter{Configuring FTP Services}

\section{Understanding FTP Configuration}
On RHEL 7, \textbf{vsftpd} is the default FTP server. The configuration files for it are stored in \verb|/etc/vsftpd| directory. The prime among these is the \verb|/etc/fsftpd/vsftpd.conf|. 

\subsection{Types of FTP users}
There can be both anonymous users and authenticated users connecting to a FTP server. Anonymous users can access the FTP site without any rights, and yet needs to access files anyway. The document root for the FTP server is \verb|/var/ftp|. It also happens to be the home directory of the system user \textit{ftp}. When an anonymous user accesses the FTP server, he gets access to the home directory of the \textit{ftp} user. However, the directory itself is owned by the user \textit{root} and belongs to the group \textit{root}. Others only have read-execute(r-x) permissions on that directory. By default, the access rights are generally configured correctly.

\section{Configuring an FTP Server for anonymous download} 
If not already available, we should ensure that the \textbf{vsftpd} service is installed and enabled with:

\vspace{-15pt}
\begin{minted}{console}
# yum install -y vsftpd
# systemctl enable vsftpd
\end{minted}
\vspace{-10pt}

\noindent
We can then list the configuration files for vsftpd using:

\vspace{-15pt}
\begin{minted}{console}
# rpm -qc vsftpd
/etc/logrotate.d/vsftpd
/etc/pam.d/vsftpd
/etc/vsftpd/ftpusers
/etc/vsftpd/user_list
/etc/vsftpd/vsftpd.conf
\end{minted}
\vspace{-10pt}

\subsection{vsftpd.conf}
The default vsftpd.conf file has an option called \verb|anonymous_enable=YES| which allows anonymous downloads. 

\noindent
\begin{tabular}{rM{0.64}}
	\toprule
	\textbf{Options} &\textbf{Description} \\
	\midrule
	\textbf{anonymous\_enable=YES}	&Allows anonymous users to download files from their own directory.\\
	\textbf{local\_enable=YES}	&Allows authenticated users to download files from their own home directory. It needs the boolean \verb|ftp_home_dir --> on| via SELinux to work. \\
	\textbf{write\_enable=YES}	&Allows authenticated users to write files to their own directory.\\
	\bottomrule
\end{tabular}

\noindent
There is nothing that needs to be changed in \verb|httpd.conf| as the default settings are configured according to our needs. Now, we can restart the \textit{vsftpd} service. 

\vspace{-15pt}
\begin{minted}{console}	
# systemctl restart vsftpd
# systemctl status vsftpd
● vsftpd.service - Vsftpd ftp daemon
Loaded: loaded (/usr/lib/systemd/system/vsftpd.service; enabled; vendor preset: disabled)
Active: active (running) since Fri 2017-12-22 15:09:30 IST; 6s ago
Process: 11516 ExecStart=/usr/sbin/vsftpd /etc/vsftpd/vsftpd.conf (code=exited, status=0/SUCCESS)
Main PID: 11517 (vsftpd)
CGroup: /system.slice/vsftpd.service
└─11517 /usr/sbin/vsftpd /etc/vsftpd/vsftpd.conf

Dec 22 15:09:30 vmPrime.somuVMnet.com systemd[1]: Starting Vsftpd ftp daemon...
Dec 22 15:09:30 vmPrime.somuVMnet.com systemd[1]: Started Vsftpd ftp daemon.

\end{minted}
\vspace{-10pt}

\noindent
We should be aware as an Admin about where the file are stored for the anonymous FTP user. This is the home directory of the FTP user, to find which we use:

\vspace{-15pt}
\begin{minted}{console}
# grep ftp /etc/passwd
ftp:x:14:50:FTP User:/var/ftp:/sbin/nologin
\end{minted}
\vspace{-10pt}

\noindent
On our system, it's configured to be on (the default directory) of \verb|/var/ftp|. Now all we need is an FTP client such as \textbf{lftp} which lets us browse the FTP directories. We can install it using \verb|yum install -y lftp|. We can connect as an anonymous ftp user using: \verb|lftp localhost|. 