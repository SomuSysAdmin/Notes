\chapter{Configuring Networking}
	\section{Understanding NIC Naming}
A Network Interface Card (NIC) is the physical hardware connecting the host machine to the network. In older versions of RHEL, the naming convention was simpler, with the Ethernet interface named simply as \textit{eth0}. In RHEL 7, there are 3 different naming schemes available.

\subsection{Network Device Naming Schemes}
\begin{tabular}{M{0.15}|M{0.4}|M{0.34}}
	\toprule
	\textbf{Scheme} &\textbf{Description} &\textbf{Example} \\
	\midrule
	\textbf{BIOS Naming} &Based on Hardware properties of the Network card. &\noindent
	\begin{minipage}{0.31\linewidth}
		\begin{tabular}{ll}
			\textbf{em[$1-N$]} &Embedded NICs\\
			\textbf{p6p5} &PCI Slot-6, Port-1\\
		\end{tabular}
	\end{minipage} \\
	\midrule
	\textbf{Udev Naming} &Classical Naming Scheme with Interface number. &\textbf{Eth$N$} - example: eth0, eth1... \\
	\midrule
	\textbf{Physical Naming} &Same as BIOS Naming & \\
	\textbf{Logical Naming} &.<vlan> and :<alias> & \\
	\bottomrule
\end{tabular}

\noindent
The entire purpose to the NIC naming scheme is to make it easier to identify the hardware NIC associated with an interface for cases where multiple NICs are available at a server. 

	\section{Managing NIC Configuration with ip Command}
An important feature of the \verb|ip| command is all data is lost with interface is reset. However, it's extremely useful as it allows us to test certain settings on NICs. The syntax of the ip command is : \verb|ip <options> <object> <command>| 

\vspace{-10pt}
\begin{center}
	\begin{tabular}{rcl}
		\toprule
		\textbf{Object} &- &\textbf{Description} \\
		\midrule
		\textbf{link} &- &Network status information \\
		\textbf{addr} &- &Set network addresses \\
		\textbf{route} &- &Helps manage routing table on the system \\
	\end{tabular}
\end{center}
\vspace{-10pt}

\noindent
To get the help for any option and object, simply replace the command with the text "help". For example, to get help about the \verb|ip addr| command, we need to type \verb|ip addr help|.

\subsection{show commands}
\subsubsection{ip link show} \vspace{-10pt}
This command displays the device attributes, and can be followed by a device name to only view the details for that device/interface. Can be used to find the available interface names and the associated MAC addresses.

\vspace{-15pt}
\begin{minted}{console}
$ ip link show
1: lo: <LOOPBACK,UP,LOWER_UP> mtu 65536 qdisc noqueue state UNKNOWN mode DEFAULT qlen 1
link/loopback 00:00:00:00:00:00 brd 00:00:00:00:00:00
2: ens33: <BROADCAST,MULTICAST,UP,LOWER_UP> mtu 1500 qdisc pfifo_fast state UP mode DEFAULT qlen 1000
link/ether 00:0c:29:d6:73:d0 brd ff:ff:ff:ff:ff:ff
3: virbr0: <NO-CARRIER,BROADCAST,MULTICAST,UP> mtu 1500 qdisc noqueue state DOWN mode DEFAULT qlen 1000
link/ether 52:54:00:a5:7f:97 brd ff:ff:ff:ff:ff:ff
4: virbr0-nic: <BROADCAST,MULTICAST> mtu 1500 qdisc pfifo_fast master virbr0 state DOWN mode DEFAULT qlen 1000
link/ether 52:54:00:a5:7f:97 brd ff:ff:ff:ff:ff:ff
\end{minted}
\vspace{-10pt}

\subsubsection{ip addr show} \vspace{-10pt}
\verb|ip addr show| command shows us the current (network) address information. Based on the interface name, we can find its specific information only as well, in which case the command has to be followed by the name of the interface. 

\vspace{-15pt}
\begin{minted}{console}
$ ip addr show ens33
2: ens33: <BROADCAST,MULTICAST,UP,LOWER_UP> mtu 1500 qdisc pfifo_fast state UP qlen 1000
link/ether 00:0c:29:d6:73:d0 brd ff:ff:ff:ff:ff:ff
inet 90.0.16.117/21 brd 90.0.23.255 scope global dynamic ens33
valid_lft 3037sec preferred_lft 3037sec
inet6 fe80::2b85:fb69:3b97:ec5f/64 scope link 
valid_lft forever preferred_lft forever
\end{minted}
\vspace{-10pt}

\noindent
Note that the \verb|inet| address is the IPv4 address whereas \verb|inet6| refers to the IPv6 address. 

\subsubsection{ip route show} \vspace{-10pt}

\vspace{-15pt}
\begin{minted}{console}
$ ip route show
default via 90.0.16.1 dev ens33 proto static metric 100 
90.0.16.0/21 dev ens33 proto kernel scope link src 90.0.16.117 metric 100 
192.168.122.0/24 dev virbr0 proto kernel scope link src 192.168.122.1 
\end{minted}
\vspace{-10pt}

\noindent
So, in this case, we have a default route which sends everything through the ip address \textit{90.0.16.1}. 

\subsection{ip addr add}
This command is used to add an IP address to a device. Note that while adding a new IP address, the address must always be followed by the Subnet Mask, as otherwise the default value of \textit{42} is applied, which doesn't make sense. 

\vspace{-15pt}
\begin{minted}{console}
# ip addr add dev ens33 10.0.0.10/24
# ip addr show ens33
2: ens33: <BROADCAST,MULTICAST,UP,LOWER_UP> mtu 1500 qdisc pfifo_fast state UP qlen 1000
link/ether 00:0c:29:75:5a:36 brd ff:ff:ff:ff:ff:ff
inet 90.0.18.206/21 brd 90.0.23.255 scope global dynamic ens33
valid_lft 3587sec preferred_lft 3587sec
inet 10.0.0.10/24 scope global ens33
valid_lft forever preferred_lft forever
inet6 fe80::ba08:1835:69e5:e9e9/64 scope link 
valid_lft forever preferred_lft forever
\end{minted}
\vspace{-10pt}

\noindent
This is one of the improvements of \verb|ip| over \verb|ifconfig| which was incapable of setting/showing multiple IP Addresses for the same device. \verb|ifconfig| is obsolete. To see the network statistics (as shown in ifconfig command), the command is :

\vspace{-15pt}
\begin{minted}{console}
# ip -s link
1: lo: <LOOPBACK,UP,LOWER_UP> mtu 65536 qdisc noqueue state UNKNOWN mode DEFAULT qlen 1
link/loopback 00:00:00:00:00:00 brd 00:00:00:00:00:00
RX: bytes  packets  errors  dropped overrun mcast   
55874      218      0       0       0       0       
TX: bytes  packets  errors  dropped carrier collsns 
55874      218      0       0       0       0       
2: ens33: <BROADCAST,MULTICAST,UP,LOWER_UP> mtu 1500 qdisc pfifo_fast state UP mode DEFAULT qlen 1000
link/ether 00:0c:29:75:5a:36 brd ff:ff:ff:ff:ff:ff
RX: bytes  packets  errors  dropped overrun mcast   
15408095   113357   0       0       0       0       
TX: bytes  packets  errors  dropped carrier collsns 
1857459    13898    0       0       0       0      
\end{minted}
\vspace{-10pt}

\subsection{ip route add}
This command is used to add a new route.

\vspace{-15pt}
\begin{minted}{console}
# ip route add 20.0.0.0/8 via 192.168.4.4
\end{minted}
\vspace{-10pt}

\noindent
However, all settings using the \verb|ip| command are temporary and thus will be reset with every reboot. To makes these settings permanent, we need to provide this info in an appropriate configuration file to be loaded during each boot.

	\section{Storing Network Configuration persistently}
The network configuration is stored in \verb|/etc/sysconfig/network-scripts| directory. There are several configuration files for each interface. The configuration file for the \textit{ens33} interface is called \verb|ifcfg-ens33|. The script looks like:

\vspace{-15pt}
\begin{minted}{bash}
TYPE=Ethernet
PROXY_METHOD=none
BROWSER_ONLY=no
BOOTPROTO=none
DEFROUTE=yes
IPV4_FAILURE_FATAL=no
IPV6INIT=yes
IPV6_AUTOCONF=yes
IPV6_DEFROUTE=yes
IPV6_FAILURE_FATAL=no
IPV6_ADDR_GEN_MODE=stable-privacy
NAME=ens33
UUID=1909bf04-c383-4aa0-afce-d774be49d3d4
DEVICE=ens33
ONBOOT=yes
HWADDR=00:0C:29:D6:73:D0
MACADDR=00:0C:29:D6:73:D0
IPADDR=192.168.4.44
PREFIX=24
GATEWAY=192.168.4.2
DNS1=8.8.8.8
\end{minted}
\vspace{-10pt}

\noindent
The BOOTPROTO can be set to \textit{dhcp} if DHCP is required. The \textit{ONBOOT="yes"} sets that the NIC should be switched on during boot. The HWADDR property specifies the MAC address. IPADDR can be specified as IPADDR0 and thus more than one ip addresses can be specified as IPADDR1, IPADDR2, and so on. 

\subsection{Hostname} 
In the \verb|/etc| directory, there is a file called \textbf{hostname} which contains the hostname of the machine we're working on. In earlier versions of RHEL, this information used to be stored in \verb|/etc/sysconfig/network/|. 

Another important file that isn't used anymore is \verb|/etc/resolv.conf|. This file is auto-generated by the Network Manager, the most important part of the system that handles network configuration. 

	\section{Understanding Network Manager}
The Network Manager is the part of the OS that manages the NIC. There are three different ways of changing the network configuration on the NIC: using the \verb|ip| command, the \verb|nmcli| command and the Network Manager TUI (Text User Interface). 

\vspace{-15pt}
\begin{minted}{console}
# ip addr add dev ens33 10.0.0.10/24 
# nmcli con add con-name ens33 ifname ens33 type ethernet ip4 10.0.0.13/24
\end{minted}
\vspace{-10pt}

When the \verb|ip addr add| command is used, the ip address is added directly to the physical NIC, which can start using it immediately. However, this data is impersistent since the information is not managed by any service. So, a reboot or even a simple bringing down of the interface erases the data. This is why the information needs to be stored in \verb|/etc/sysconfig/network-scripts/ifcfg-ens33|. 

So, when either the \verb|nmcli| or the Network Manager TUI is used to configure the network settings, the Network Manager service ensures the data is stored in the above file and is thus available after every boot or interface restart. 

The \verb|ip| command is used for temporary changes only, while the Network Manager is used for persistent changes.

 	\section{Using Network Manager utilities (nmcli, nmtui)}
The \verb|nmcli| tool controls the Network Manager from the command line. Just like the \verb|ip| command, its syntax is \verb|nmcli <options> <object> <command>|. The objects are like subcommands as in the case of \verb|ip| command. The most important object is the \textbf{connection(c)}.

A network interface is just an interface with some connection associated with it. A connection is however, an abstract layer lying on top of the interface. The concept is that every Network interface has a default connection, but we can add a testing connection as well. We can then switch between these connections using the network manager utilities. 

\subsection{nmcli}
\vspace{-5pt}
\subsubsection{nmcli connection show}
\vspace{-15pt}
\begin{minted}{console}
$ nmcli connection show
NAME   UUID                                  TYPE            DEVICE 
ens33  26c54678-6784-47ba-8afc-ca9924ed63af  802-3-ethernet  ens33
\end{minted}
\vspace{-10pt}

\noindent
The command to add a new connection is: (line 9)

\vspace{-15pt}
\begin{minted}{console}
# ls /etc/sysconfig/network-scripts/
ifcfg-ens33  ifdown-eth   ifdown-post    ifdown-Team      ifup-aliases  ifup-ipv6   ifup-post    ifup-Team      init.ipv6-global
ifcfg-lo     ifdown-ippp  ifdown-ppp     ifdown-TeamPort  ifup-bnep     ifup-isdn   ifup-ppp     ifup-TeamPort  network-functions
ifdown       ifdown-ipv6  ifdown-routes  ifdown-tunnel    ifup-eth      ifup-plip   ifup-routes  ifup-tunnel    network-functions-ipv6
ifdown-bnep  ifdown-isdn  ifdown-sit     ifup             ifup-ippp     ifup-plusb  ifup-sit     ifup-wireless
# nmcli connection show
NAME   UUID                                  TYPE            DEVICE 
ens33  26c54678-6784-47ba-8afc-ca9924ed63af  802-3-ethernet  ens33  
# nmcli con add con-name testing ifname ens33 type ethernet ip4 10.0.0.15/24
Connection 'testing' (8bc5959e-3d1e-4738-8fa6-b584e4ba4388) successfully added.
# nmcli connection show
NAME     UUID                                  TYPE            DEVICE 
ens33    26c54678-6784-47ba-8afc-ca9924ed63af  802-3-ethernet  ens33  
testing  8bc5959e-3d1e-4738-8fa6-b584e4ba4388  802-3-ethernet  --     
# ls /etc/sysconfig/network-scripts/
ifcfg-ens33    ifdown-bnep  ifdown-isdn    ifdown-sit       ifup          ifup-ippp  ifup-plusb   ifup-sit       ifup-wireless
ifcfg-lo       ifdown-eth   ifdown-post    ifdown-Team      ifup-aliases  ifup-ipv6  ifup-post    ifup-Team      init.ipv6-global
ifcfg-testing  ifdown-ippp  ifdown-ppp     ifdown-TeamPort  ifup-bnep     ifup-isdn  ifup-ppp     ifup-TeamPort  network-functions
ifdown         ifdown-ipv6  ifdown-routes  ifdown-tunnel    ifup-eth      ifup-plip  ifup-routes  ifup-tunnel    network-functions-ipv6
\end{minted}
\vspace{-10pt}

\noindent
Note that upon creation of the new connection, the \verb|nmcli| tool also creates a config file called \verb|/etc/sysconfig/network-scripts/ifcfg-testing| (line 18) for the new connection called \textit{testing}.

\subsubsection{Switching Connections}
The connection switching is as simple as bringing an interface down and bringing an alternative up. We do this using \verb|nmcli| by:
\vspace{-15pt}
\begin{minted}{console}
# nmcli con show
NAME     UUID                                  TYPE            DEVICE 
ens33    26c54678-6784-47ba-8afc-ca9924ed63af  802-3-ethernet  ens33  
testing  8bc5959e-3d1e-4738-8fa6-b584e4ba4388  802-3-ethernet  --     
# nmcli con down ens33
Connection 'ens33' successfully deactivated (D-Bus active path: /org/freedesktop/NetworkManager/ActiveConnection/11)
# nmcli con up testing
Connection successfully activated (D-Bus active path: /org/freedesktop/NetworkManager/ActiveConnection/13)
# nmcli con show
NAME     UUID                                  TYPE            DEVICE 
testing  8bc5959e-3d1e-4738-8fa6-b584e4ba4388  802-3-ethernet  ens33  
ens33    26c54678-6784-47ba-8afc-ca9924ed63af  802-3-ethernet  --     
\end{minted}
\vspace{-10pt}

\subsection{nmtui}
This command provides a Text User Interface for the Network Manager. 

On restarting the Network Manager, all the set connections become simultaneously activated. To avoid this, we would have to bring the connection down on restart either through the config files or the TUI. 

\vspace{-15pt}
\begin{minted}{console}
# systemctl restart NetworkManager
# systemctl status -l NetworkManager
\end{minted}
\vspace{-10pt}

	\section{Understanding Routing and DNS}
\subsection{Default route}
To connect to another network (or the internet), the server needs to know an IP Address of another computer that can connect it to the desired network. This is called the \textbf{default route}. It must be present on the same network as the host. 

If we consider in the diagram that the Computer with the IP address \textit{192.168.1.19} is our host that needs a packet to reach another computer on the internet, then the default route for it would be \textit{192.68.1.1} as it is the computer through which our server is connected to another network. Further, to pass along the packet to the receiver on the internet, the computer with IP \textit{192.168.1.1} will have to go through another computer's IP as it's own default route that can connect it to the internet. Here, that is \textit{10.0.0.1}.

\begin{figure}[H]
	\centering
	\includegraphics[height=0.9\linewidth]{"RHCSA/Mod1/chapters/1.9.a Default Route"}
	\caption{Default Route}
	\label{fig:1}
\end{figure}
\vspace{-5pt}

Let us consider another scenario where the host \textit10.0.0.20{} wants to send a packet to \textit{192.168.1.19}. Its default route will have to through \textit{192.168.1.1}. 

Now, consider \textit{10.0.0.20} needs to send a packet back to \textit{192.168.1.19}, but also needs to send packets to the internet. Then, the default route would be the IP address of the computer that can connect us to the internet (\textit{10.0.0.1}). However,  to eventually reach \textit{192.168.1.19}, the packets need a way to reach \textit{192.168.1.1} first, given by \textit{10.0.0.10}. Thus, sometimes a computer needs to be configured for multiple routers.

\subsection{DNS}
A Domain Name System (DNS) server stores the domain names along with a list of their corresponding IP addresses, and when packets destined for a certain domain name are available, it provides the actual IP addresses for it. It translates the domain name to an IP address.

\section{Configuring Routing and DNS} 
On a temporary basis, the \verb|ip route add| command can add a new default route. The settings can be shown using \verb|ip route show|. However, to make the settings permanent, we need to edit the file \verb|/etc/sysconfig/network-scripts/ifcfg-ens33| (where \textit{ens33} is the name of our interface). Merely changing the value of the Gateway will suffice. After changing the value of the gateway, we need to bring the interface down and up again!

\vspace{-15pt}
\begin{minted}{console}
# nmcli con down ens33; nmcli con up ens33
Connection 'ens33' successfully deactivated (D-Bus active path: /org/freedesktop/NetworkManager/ActiveConnection/3)
Connection successfully activated (D-Bus active path: /org/freedesktop/NetworkManager/ActiveConnection/5)
\end{minted}
\vspace{-10pt}

Likewise, the values for DNS Server(s) are also specified in the \verb|ifcfg-ens33| file. There can be multiple DNS servers, where the successive server(s) are contacted (in order) only if the preceding ones were down (or couldn't be contacted). 

	\section{Understanding Network Analysis Tools}
The following are a few network analysis tools that help diagnose problems with the network. 

\begin{tabular}{rM{0.77}}
	\toprule
	\textbf{Name} &\textbf{Description} \\
	\midrule
	\textbf{hostname} &Shows current hostname and provides an option to change it. \\
	\textbf{ping} &Performs a connectivity test to know if another computer can be reached. \\
	\textbf{traceroute} &Provides specific information about the routing between the host and a destination computer. \textit{NOTE} that many routers are configured nowadays to not display information about their operation, and thus information from \verb|traceroute| may be inaccurate.\\
	\textbf{dig} &Shows DNS information and helps diagnose DNS related problems.\\
	\textbf{nmap} &Advanced and potentially dangerous tool to get information about remote service availability. Since a portscan can be performed to determine which services are provided by a server using it, this utility is considered hostile by many NetAdmins. \\
	\textbf{netscan} 
	&\begin{minipage}{0.75\linewidth}
		\begin{tabular}{lM{0.95}}
			\toprule
			\textbf{Command} &\textbf{Description} \\
			\midrule
			\textbf{netstat -i} &Packet information for network cards. \\
			\textbf{netstat -tulpen} &Information about listening ports on a server: \\
			&\begin{tabular}{rcl}
				\textbf{t}	&- &TCP \\
				\textbf{u}	&- &UDP \\
				\textbf{l}	&- &Listening \\
				\textbf{p}	&- &Process Information \\
				\textbf{e}	&- &Extended Information \\
				\textbf{n}	&- &Names \\
			\end{tabular} \\
			%\bottomrule
		\end{tabular}
	\end{minipage} 	\\
	\bottomrule
\end{tabular}

	\section{Using Network Analysis Tools}
For troubleshooting network issues, we first check our own network information. We use \verb|ip addr show| to ensure our IP address is correct. Then, we use \verb|ip route show| to ensure that the default route is set to an IP that's in the same IP network as (one of) our own IP address. Next we check the DNS name resolution by printing \verb|/etc/resolv.conf|. 

\vspace{-15pt}
\begin{minted}{console}
# ip addr show
# ip route show
# cat /etc/resolv.conf
\end{minted}
\vspace{-10pt}

\subsection{ping}
If the problem still persists, further testing is required. First we ping the nearest router, then the one after that till we can reach the internet, unless there's an error, in which case we can know exactly which network has a problem. 

\vspace{-15pt}
\begin{minted}{console}
# ping -c 1 192.168.0.1 # Pinging the default router by sending 1 packet. 
\end{minted}
\vspace{-10pt}

\subsubsection{ping flood test}
This is a bandwidth test, instead of a connectivity test and tells us how many packets are being dropped on real time. 

\vspace{-15pt}
\begin{minted}{console}
# ping -f cliServer
PING cliServer.somuVMnet.local (90.0.18.206) 56(84) bytes of data.
.^ 
--- cliServer.somuVMnet.local ping statistics ---
5215 packets transmitted, 5215 received, 0% packet loss, time 3354ms
rtt min/avg/max/mdev = 0.140/0.381/6.757/0.516 ms, ipg/ewma 0.643/0.287 ms
\end{minted}
\vspace{-10pt}

\noindent
It prints a \verb|.| for every ECHO\_REQUEST and prints a backspace for every reply. Thus, the frequency of appearance of \verb|.|s on the screen represents the frequency of packet loss. Further, since an interval is not given, it sends the packets as soon as a reply is received, thus giving us a measure of the bandwidth of the line in use. NOTE that superuser privileges are required to flood ping without an interval. 

\subsection{traceroute}
En-route to the destination server, the time it took to reach every router and the time taken is displayed by this command. If there is some kind of filtering enabled on the server, then the data is redacted with \verb|*|s.  

\vspace{-15pt}
\begin{minted}{console}
# traceroute 8.8.8.8
traceroute to 8.8.8.8 (8.8.8.8), 30 hops max, 60 byte packets
1  amontpellier-653-1-352-1.w90-0.abo.wanadoo.fr (90.0.16.1)  4.277 ms  4.078 ms  3.954 ms
2  202.38.180.121 (202.38.180.121)  7.041 ms  6.943 ms  6.890 ms
3  10.10.50.1 (10.10.50.1)  6.796 ms  6.565 ms  6.614 ms
4  202.38.180.50 (202.38.180.50)  12.069 ms  11.962 ms  11.829 ms
5  108.170.253.97 (108.170.253.97)  18.954 ms  18.886 ms 108.170.253.113 (108.170.253.113)  18.768 ms
6  209.85.240.133 (209.85.240.133)  18.232 ms 209.85.240.159 (209.85.240.159)  12.603 ms 72.14.239.7 (72.14.239.7)  12.381 ms
7  google-public-dns-a.google.com (8.8.8.8)  12.223 ms  15.113 ms  14.561 ms
\end{minted}
\vspace{-10pt}

\subsection{host}
The host command returns the IP address of any domain name that the machine can resolve with the host-name resolution process. This means a result will be produced even if the host is merely added in the \verb|/etc/hosts| file and not configured on the DNS server. 

\vspace{-15pt}
\begin{minted}{console}
# host www.somusysadmin.com
www.somusysadmin.com has address 104.27.137.245
www.somusysadmin.com has address 104.27.136.245
www.somusysadmin.com has IPv6 address 2400:cb00:2048:1::681b:89f5
www.somusysadmin.com has IPv6 address 2400:cb00:2048:1::681b:88f5
\end{minted}
\vspace{-10pt}

\subsection{dig}
\textbf{dig} gives more in-depth information about the name lookup process through DNS. This means no information is provided if the destination machine is unknown to the DNS server. 

\vspace{-15pt}
\begin{minted}{console}
# dig www.somusysadmin.com

; <<>> DiG 9.9.4-RedHat-9.9.4-51.el7 <<>> www.somusysadmin.com
;; global options: +cmd
;; Got answer:
;; ->>HEADER<<- opcode: QUERY, status: NOERROR, id: 65365
;; flags: qr rd ra; QUERY: 1, ANSWER: 2, AUTHORITY: 13, ADDITIONAL: 12

;; QUESTION SECTION:
;www.somusysadmin.com.		IN	A

;; ANSWER SECTION:
www.somusysadmin.com.	199	IN	A	104.27.136.245
www.somusysadmin.com.	199	IN	A	104.27.137.245

;; AUTHORITY SECTION:
com.			100319	IN	NS	a.gtld-servers.net.
com.			100319	IN	NS	j.gtld-servers.net.
com.			100319	IN	NS	c.gtld-servers.net.
com.			100319	IN	NS	m.gtld-servers.net.
com.			100319	IN	NS	l.gtld-servers.net.
com.			100319	IN	NS	h.gtld-servers.net.
com.			100319	IN	NS	b.gtld-servers.net.
com.			100319	IN	NS	d.gtld-servers.net.
com.			100319	IN	NS	i.gtld-servers.net.
com.			100319	IN	NS	g.gtld-servers.net.
com.			100319	IN	NS	k.gtld-servers.net.
com.			100319	IN	NS	f.gtld-servers.net.
com.			100319	IN	NS	e.gtld-servers.net.

;; ADDITIONAL SECTION:
a.gtld-servers.net.	34680	IN	A	192.5.6.30
j.gtld-servers.net.	97045	IN	A	192.48.79.30
c.gtld-servers.net.	88438	IN	A	192.26.92.30
m.gtld-servers.net.	88436	IN	A	192.55.83.30
l.gtld-servers.net.	32537	IN	A	192.41.162.30
h.gtld-servers.net.	88436	IN	A	192.54.112.30
b.gtld-servers.net.	88438	IN	A	192.33.14.30
d.gtld-servers.net.	48310	IN	A	192.31.80.30
i.gtld-servers.net.	99400	IN	A	192.43.172.30
g.gtld-servers.net.	88436	IN	A	192.42.93.30
f.gtld-servers.net.	143261	IN	A	192.35.51.30
e.gtld-servers.net.	138671	IN	A	192.12.94.30

;; Query time: 766 msec
;; SERVER: 8.8.8.8#53(8.8.8.8)
;; WHEN: Wed Nov 29 11:59:42 IST 2017
;; MSG SIZE  rcvd: 486
\end{minted}
\vspace{-10pt}

\noindent
The status of \verb|NOERROR| indicates the operation was successful. The question section containing \verb|www.somusysadmin.com.	IN	A| indicates that an address for \verb|www.somusysadmin.com| was queried and the \verb|A| indicates an address was asked for. 

The answer section is provided with the different IP Addresses for the domain name. At the bottom, the server that was queried and operation details are noted.

\verb|SERVER: 8.8.8.8#53(8.8.8.8)|.

To perform a bit of performance optimization, we can add our own name server before a public DNS if we want to directly fetch the data. To do this, simply add a new DNS in \verb|/etc/sysconfig/network-scripts/ifcfg-ens33|. After this, a restart of the NetworkManager is required for the new configuration settings to take effect. f

\vspace{-15pt}
\begin{minted}{console}
# systemctl restart NetworkManager
\end{minted}
\vspace{-10pt}

When we try to dig a site that doesn't exist, the output is :

\vspace{-15pt}
\begin{minted}{console}
# dig siteDoesntExist.com

; <<>> DiG 9.9.4-RedHat-9.9.4-51.el7 <<>> siteDoesntExist.com
;; global options: +cmd
;; Got answer:
;; ->>HEADER<<- opcode: QUERY, status: NXDOMAIN, id: 22389
;; flags: qr rd ra; QUERY: 1, ANSWER: 0, AUTHORITY: 0, ADDITIONAL: 0

;; QUESTION SECTION:
;sitedoesntexist.com.		IN	A

;; Query time: 2 msec
;; SERVER: 8.8.8.8#53(8.8.8.8)
;; WHEN: Wed Nov 29 12:27:15 IST 2017
;; MSG SIZE  rcvd: 37
\end{minted}
\vspace{-10pt}

The \verb|NXDOMAIN| status is indicative of the fact that the domain is non-existent. 

\subsection{Physical network problems}
Sometimes there may be a physical problem in the network and not a problem with the configuration. In that case, the \verb|ip -s link| command can give us a hint about the statistics of the interface and thus show us if packets are being dropped, etc.

\vspace{-15pt}
\begin{minted}{console}
# ip -s link
1: lo: <LOOPBACK,UP,LOWER_UP> mtu 65536 qdisc noqueue state UNKNOWN mode DEFAULT qlen 1
link/loopback 00:00:00:00:00:00 brd 00:00:00:00:00:00
RX: bytes  packets  errors  dropped overrun mcast   
126271     1008     0       0       0       0       
TX: bytes  packets  errors  dropped carrier collsns 
126271     1008     0       0       0       0       
2: ens33: <BROADCAST,MULTICAST,UP,LOWER_UP> mtu 1500 qdisc pfifo_fast state UP mode DEFAULT qlen 1000
link/ether 00:0c:29:d6:73:d0 brd ff:ff:ff:ff:ff:ff
RX: bytes  packets  errors  dropped overrun mcast   
22610122   182819   0       0       0       0       
TX: bytes  packets  errors  dropped carrier collsns 
1686685    16970    0       0       0       0       
3: virbr0: <NO-CARRIER,BROADCAST,MULTICAST,UP> mtu 1500 qdisc noqueue state DOWN mode DEFAULT qlen 1000
link/ether 52:54:00:a5:7f:97 brd ff:ff:ff:ff:ff:ff
RX: bytes  packets  errors  dropped overrun mcast   
0          0        0       0       0       0       
TX: bytes  packets  errors  dropped carrier collsns 
0          0        0       0       0       0       
4: virbr0-nic: <BROADCAST,MULTICAST> mtu 1500 qdisc pfifo_fast master virbr0 state DOWN mode DEFAULT qlen 1000
link/ether 52:54:00:a5:7f:97 brd ff:ff:ff:ff:ff:ff
RX: bytes  packets  errors  dropped overrun mcast   
0          0        0       0       0       0       
TX: bytes  packets  errors  dropped carrier collsns 
0          0        0       0       0       0       
\end{minted}
\vspace{-10pt}