\chapter{Applying Essential Troubleshooting Skills}

\section{Making Grub Changes persistent}
\subsection{Changes made during boot}
After making changes in the boot menu, when we finally boot, we can make those changes persistent by rewriting the \verb|/boot/grub2/grub.cfg| file. 

\vspace{-15pt}
\begin{minted}{console}
# grub2-mkconfig -o /boot/grub2/grub.cfg
Generating grub configuration file ...
Found linux image: /boot/vmlinuz-3.10.0-693.el7.x86_64
Found initrd image: /boot/initramfs-3.10.0-693.el7.x86_64.img
Found linux image: /boot/vmlinuz-0-rescue-5cbfb880c0aa466ca7e3be91308fde5f
Found initrd image: /boot/initramfs-0-rescue-5cbfb880c0aa466ca7e3be91308fde5f.img
done
\end{minted}
\vspace{-10pt}

\subsection{Changes made in Configuraiton File}
The \verb|/etc/default/grub| file is the configuration file for Grub2 that provides several boot options. These can be changed to affect several boot parameters, and the changes saved to the bootloader. There are also shell scripts in the \verb|/etc/grub.d| directory that aren't meant to be touched by an administrator. These control grub boot procedure as well. Almost all the functionality that we need from grub is provided by a set of grub2 commands:

\vspace{-15pt}
\begin{minted}{console}
# grub2-
grub2-bios-setup           grub2-mkpasswd-pbkdf2
grub2-editenv              grub2-mkrelpath
grub2-file                 grub2-mkrescue
grub2-fstest               grub2-mkstandalone
grub2-get-kernel-settings  grub2-ofpathname
grub2-glue-efi             grub2-probe
grub2-install              grub2-reboot
grub2-kbdcomp              grub2-rpm-sort
grub2-menulst2cfg          grub2-script-check
grub2-mkconfig             grub2-set-default
grub2-mkfont               grub2-setpassword
grub2-mkimage              grub2-sparc64-setup
grub2-mklayout             grub2-syslinux2cfg
grub2-mknetdir             
\end{minted}
\vspace{-10pt}

\noindent
These commands can be used to accomplish tasks with grub such as install grub (\verb|grub2-install|), make a new boot image (\verb|grub2-mkimage|), set a grub boot password (\verb|grub2-mkpasswd-pbkdf2|), to probe operating system configuration (\verb|grub2-probe|), to reboot a specific boot image (\verb|grub2-reboot|) and much more. 

\section{Using rd.break to Reset the Root Password}
While on the previous versions of RHEL, resetting the root password or logging on to a system where the root password isn't know was relatively easy. After the introduction of systemd, breaking into the system is a lot harder to do. 

First we have to enter the line \verb|rd.break| and pass it as a kernel parameter in the boot menu (at the end of the kernel line). The \textbf{rd.break} parameter instructs the next part of the boot procedure, \textbf{initrd}, to break at a specific location of the \textit{initramfs}. This brings us to a system where all the supporting modules are available, but no file system has yet been mounted. This parameter bring us to a root shell without prompting for a root password. 

We're in such an early point in the boot procedure that the system root hasn't been mounted to the usual \verb|/| location yet, and is available at \verb|/sysroot| in read-only mode. Now, we need to mount the system root in a read-write mode using:

\vspace{-15pt}
\begin{minted}{console}
# mount -o remount,rw /sysroot
\end{minted}
\vspace{-10pt}

\noindent
Next, we make the content of \verb|/sysroot| the current root directory using:

\vspace{-15pt}
\begin{minted}{console}
# chroot /sysroot
\end{minted}
\vspace{-10pt}

\noindent
Now, we simply echo the new password to the passwd utility and reset the password for the user root. The syntax is: \verb:echo <newPassword> | passwd --stdin root:. The root password thus has to be reset using the command:

\vspace{-15pt}
\begin{minted}{console}
# echo secret | passwd --stdin root
Changing password for the user root.
passwd: all authentication tokens updated successfully. 
# touch /.autorelabel
\end{minted}
\vspace{-10pt}

\noindent
Finally, in the last line, we instruct SELinux to auto-relable. Since we're so early in the boot procedure, SELinux isn't functional, and if we skip this command, our changes will be lost. Now, at this point, it is safe to \textit{CTRL+D} a couple of times and let the OS reboot itself. Once done, we can enter the OS using the root password we just set (\textit{secret} in our case). Now, after the reboot, we can login to the system as root using the new root password. 

