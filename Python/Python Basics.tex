\documentclass{report}
\usepackage[utf8]{inputenc}

%	Changing document font to Helvetica.
\usepackage[scaled]{helvet}
\renewcommand\familydefault{\sfdefault} 
\usepackage[T1]{fontenc}

%	Changing Margins and other formatting
\usepackage{geometry}
\geometry{
	a4paper,
	total={170mm,257mm},
	left=1.5in,
	top=1in,
	right=1.5in,
	bottom=1in
}
\setlength{\parskip}{1em}

%	Source Code Highlighting
\usepackage{minted}
%	For Python Code
\setminted[python]{
frame=lines,
framesep=2mm,
baselinestretch=1.2,
fontsize=\footnotesize,
linenos,
breaklines
}
%	For Output
\newminted{pycon}{
bgcolor=bg, 
linenos=true, 
tabsize=4
}
%	Pycon formatting:
\setminted[pycon]{
frame=lines,
framesep=2mm,
baselinestretch=1.2,
fontsize=\footnotesize,
breaklines
}

%	Pretty Tables
\usepackage{booktabs}
\usepackage{array, multirow}

%	Custom column for tables
\newcolumntype{P}[1]{ >{\centering\arraybackslash} m{#1\linewidth} }
\newcolumntype{M}[1]{m{#1\linewidth}}

%	Images Support
\usepackage{graphicx}

%	Support for spaces in file names
\usepackage[space]{grffile}

\title{Python Basics}
\author{Somenath Sinha}
\date{December 2017}

\begin{document}
	\maketitle
	\tableofcontents
	\chapter{Basics}
	\section{Syntax}
	The lines in python don't end with semicolon. Thus, the end of the lines matter, and the spaces matter. The 
	\section{Printing}
	
	Printing is done with \verb|print()|. Each print automatically prints a newline at end, unless the end character is specified. The escape sequences are respected as usual. 
	\vspace{-15pt}
	\begin{minted}{python}
	print("Hello")
	print("Hello")
	print("Hello", end="")
	print("Hello")
	print("Hello", end="! ")
	print("Hello")
	print("Hel\nlo")
	print("Hello")
	print("He\tllo")
	print("Hello")
	\end{minted}
	\vspace{-10pt}
	
	\textit{Output}
	\vspace{-15pt}
	\begin{minted}{pycon}
	Hello
	Hello
	HelloHello
	Hello! Hello
	Hel
	lo
	Hello
	He	llo
	Hello
	\end{minted}
	\vspace{-10pt}
	
	\subsection{Printing multiple words}
	The procedure to print multiple words using the same print is:
	
	\vspace{-15pt}
	\begin{minted}{python}
	print("Hello"+"World")
	\end{minted}
	\vspace{-10pt}
	
	\textit{Output}
	
	\vspace{-15pt}
	\begin{minted}{pycon}
	HelloWorld
	\end{minted}
	\vspace{-10pt}
	
	\subsection{Comments}
	\vspace{-15pt}
	\begin{minted}{python}
	#   Single line comment
	
	'''
	Block comment!
	'''
	\end{minted}
	\vspace{-10pt}
	
	\subsection{Variables}
	Python doesn't need a specific datatype declaration. So, we can directly assign a value to a variable. In python, both single and double quotes represent a string. 
	
	\vspace{-15pt}
	\begin{minted}{python}
	x = 4.5
	y = 'a word'
	z = "a new string"
	\end{minted}
	\vspace{-10pt}
	
	\subsection{Printing Value of a variable}
	When printing a variable, python automatically prints a space every time a variable's value is printed. 
	
	\vspace{-15pt}
	\begin{minted}{python}
	x = 4
	print("x =",x)
	y = "Space"
	print("Without", y, sep="")
	\end{minted}
	\vspace{-10pt}
	
	\textit{Output}
	\vspace{-15pt}
	\begin{minted}{pycon}
	x = 4
	WithoutSpace
	\end{minted}
	\vspace{-10pt}
	
	\section{Arithmetic Operations}
	\subsection{Basic Arithmetic}
	In Python +, -, * and \% (modulus) all act as in Java. Division however acts different. 
	
	\vspace{-15pt}
	\begin{minted}{python}
	a=5
	b=4
	x=a+b
	y=a-b
	z=a*b
	w=a%b
	
	print("a+b =",x)
	print("a-b =",y)
	print("a*b =",z)
	print("a%b =",w)
	\end{minted}
	\vspace{-10pt}
	
	\textit{Output}
	\vspace{-15pt}
	\begin{minted}{pycon}
	a+b = 9
	a-b = 1
	a*b = 20
	a%b = 1
	\end{minted}
	\vspace{-10pt}
	
	\subsubsection{Division}
	In case of java, the division is called integer division where integer truncation occurs with the result. In python, a value with a decimal point will be returned. To bypass this, we use the \verb|//| (floor division) operator. 
	
	\vspace{-15pt}
	\begin{minted}{python}
	a=5
	b=4
	x=a/b
	y=a//b
	
	
	print("a/b =",x)
	print("a//b =",y)
	\end{minted}
	\vspace{-10pt}
	
	\textit{Output}
	\vspace{-15pt}
	\begin{minted}{pycon}
	a/b = 1.25
	a//b = 1
	\end{minted}
	\vspace{-10pt}
	
	\subsubsection{Exponents}
	The exponent operator is \verb|**|.
	
	\vspace{-15pt}
	\begin{minted}{python}
	x=2**3
	
	print("2^3 =",x)
	\end{minted}
	\vspace{-10pt}
	
	\textit{Output}
	\vspace{-15pt}
	\begin{minted}{pycon}
	2^3 = 8
	\end{minted}
	\vspace{-10pt}
	
	\subsection{Casting}
	\vspace{-15pt}
	\begin{minted}{python}
	x=int(3.5)
	
	print("x =",x)
	\end{minted}
	\vspace{-10pt}
	
	\textit{Output}
	\vspace{-15pt}
	\begin{minted}{pycon}
	x = 3
	\end{minted}
	\vspace{-10pt}

	\subsection{Library Math functions}
	\vspace{-15pt}
	\begin{minted}{python}
	x=max(3,4,5)
	y=min(3,4,5)
	
	print("Max =",x)
	print("Min =",y)
	\end{minted}
	\vspace{-10pt}
	
	\textit{Output}
	\vspace{-15pt}
	\begin{minted}{pycon}
	Max = 5
	Min = 3
	\end{minted}
	\vspace{-10pt}
	
	\section{User Input}
	\vspace{-15pt}
	\begin{minted}{python}
	print("Enter a value for x: ")
	x=input();
	y=input("Test value for y: ")
	print("x =",x)
	print("y =",y)
	\end{minted}
	\vspace{-10pt}
	
	\textit{Output}
	\vspace{-15pt}
	\begin{minted}{pycon}
	Enter a value for x: 
	5
	Test value for y: 6
	x = 5
	y = 6
	\end{minted}
	\vspace{-10pt}
	
	\subsection{Casting user input}
	If the input needs to be casted, it should be done so immediately, after the input.
	
	\vspace{-15pt}
	\begin{minted}{python}
	x=float(input("Enter a num: "))
	y=int(x)
	print("x =", x)
	print("y =", y)
	\end{minted}
	\vspace{-10pt}
	
	\textit{Output}
	\vspace{-15pt}
	\begin{minted}{pycon}
	Enter a num: 3.5
	x = 3.5
	y = 3
	\end{minted}
	\vspace{-10pt}
	
	\section{String functions}
	Just like in Java, strings are immutable in Python, and thus each string function returns a new string. 
	\subsection{Printing String length}
	\vspace{-15pt}
	\begin{minted}{python}
	s="input"
	print("Length of s =",len(s))
	\end{minted}
	\vspace{-10pt}
	
	\textit{Output}
	\vspace{-15pt}
	\begin{minted}{pycon}
	Length of s = 5
	\end{minted}
	\vspace{-10pt}
	
	\subsection{Substring}
	\vspace{-15pt}
	\begin{minted}{python}
	s="input"
	print("Last 3 characters: ",s[2:])	# Called slice notation
	print("2rd and 4th characters: ",s[2:4])
	print("Last 3 characters: ", s[-2:])	# Negative index indicates count from the last.
	\end{minted}
	\vspace{-10pt}
	
	\textit{Output}
	\vspace{-15pt}
	\begin{minted}{pycon}
	Last 3 characters:  put
	2rd and 4th characters:  pu
	Last 3 characters:  ut
	\end{minted}
	\vspace{-10pt}
	
	\subsection{In operator}
	\vspace{-15pt}
	\begin{minted}{python}
	s="input"
	print("Contains pu: ", "pu" in s)   # Returns true if the string is present in s.
	\end{minted}
	\vspace{-10pt}
	
	\textit{Output}
	\vspace{-15pt}
	\begin{minted}{pycon}
	Contains pu:  True
	\end{minted}
	\vspace{-10pt}
	
	\chapter{Datastructures}
	\section{Lists}
	\vspace{-15pt}
	\begin{minted}{python}
	list = []   # Creates an empty list
	list.append("House")
	list.append("Mouse")
	list.append("Blouse")
	print(list)
	print("Size :", len(list))
	print("Index 1:", list[1])
	list.insert(1,"Grouse")
	print("Index 1:", list[1], "\nEntire list:", list)
	del(list[1:2])    # Delete the item at index 1 & 2 of list
	print(list)
	\end{minted}
	\vspace{-10pt}
	
	\textit{Output}
	\vspace{-15pt}
	\begin{minted}{pycon}
	['House', 'Mouse', 'Blouse']
	Size : 3
	Index 1: Mouse
	Index 1: Grouse 
	Entire list: ['House', 'Grouse', 'Mouse', 'Blouse']
	['House', 'Blouse']
	\end{minted}
	\vspace{-10pt}
	
	\subsection{Extend}
	\vspace{-15pt}
	\begin{minted}{python}
	list1 = []   # Creates an empty list1
	list1.append("House")
	list1.append("Mouse")
	list1.append("Blouse")
	
	list2 = []   # Creates an empty list2
	list2.append("House")
	list2.append("Mouse")
	list2.append("Blouse")
	
	list3 = list1 + list2
	list1.extend(list2)
	
	print(list1)
	print(list3)
	
	list4 = []  # Will be a list of lists!
	list4.append(list1)
	list4.append(list2)
	print(list4)
	print(list4[0][4])
	\end{minted}
	\vspace{-10pt}
	
	\textit{Output}
	\vspace{-15pt}
	\begin{minted}{pycon}
	['House', 'Mouse', 'Blouse', 'House', 'Mouse', 'Blouse']
	['House', 'Mouse', 'Blouse', 'House', 'Mouse', 'Blouse']
	[['House', 'Mouse', 'Blouse', 'House', 'Mouse', 'Blouse'], ['House', 'Mouse', 'Blouse']]
	Mouse
	\end{minted}
	\vspace{-10pt}
	
	\subsection{Immutable Tuples}
	A \verb|tuple| is an immutable list. Once created, it cannot be changed, although new tuples can be created from it. 
	\vspace{-15pt}
	\begin{minted}{python}
	x= 2,3,4,5  # x is a tuple.
	print(x)
	\end{minted}
	\vspace{-10pt}
	
	\textit{Output}
	\vspace{-15pt}
	\begin{minted}{pycon}
	(2, 3, 4, 5)
	\end{minted}
	\vspace{-10pt}
	
	\subsubsection{Tuple of tuples}
	\vspace{-15pt}
	\begin{minted}{python}
	x= 2,3,4,5  # x is a tuple.
	print(x)
	y= x,6,7
	print(y)
	\end{minted}
	\vspace{-10pt}
	
	\textit{Output}
	\vspace{-15pt}
	\begin{minted}{pycon}
	(2, 3, 4, 5)
	((2, 3, 4, 5), 6, 7)
	\end{minted}
	\vspace{-10pt}
	
	\subsubsection{Tuple of only 1 element}
	\vspace{-15pt}
	\begin{minted}{python}
	x=5,
	print(x)
	\end{minted}
	\vspace{-10pt}
	
	\textit{Output}
	\vspace{-15pt}
	\begin{minted}{pycon}
	(5,)
	\end{minted}
	\vspace{-10pt}
	
	\subsection{Tuple functions}
	\vspace{-15pt}
	\begin{minted}{python}
	x=5,4,3,2,1
	print(x)
	print(len(x))
	print(3 in x)
	\end{minted}
	\vspace{-10pt}
	
	\textit{Output}
	\vspace{-15pt}
	\begin{minted}{pycon}
	(5, 4, 3, 2, 1)
	5
	True
	\end{minted}
	\vspace{-10pt}
	
\end{document}