\chapter{Managing Time Services}
	
\section{Understanding RHEL7 Time Services}
The command to control the current time and date is called \verb|timedatectl|. For example, to set the current time, we can use \verb|timedatectl set-time 9:00:00|. To use a NTP server, the \textbf{chronyd} service is used. To start using a NTP server, we use:

\vspace{-15pt}
\begin{minted}{console}
# timedatectl set-ntp true
\end{minted}
\vspace{-10pt}	

\noindent
The configuration file for the chronyd service is : \verb|/etc/chrony.conf|. By default the NTP server used is \textit{ntp.pool.org}. 

\section{Configuring NTP Peers}
\subsection{Server vs Peer}
The difference between an NTP server and peer is authority. A server is always authoritative, i.e., whatever time the NTP server says it is, our server will set it's clocks accordingly \textbf{if withing tolerances}, which is \textit{1000 seconds} by default, as defined by the NTP protocol. 

NTP peers are not authoritative. If two peers have different times set, they'll try to \textit{middle-time}, i.e., they'll split the difference in time. So, if \textit{peer1} thinks it's 8:00AM and \textit{peer2} thinks it's 9:00AM, they'll split the difference and set the clocks to \textit{8.30AM}. 

While servers are always better, peers provide redundancy and ensure that time configuration doesn't go haywire if the internet connection drops out.

\vspace{-10pt}
\subsection{Peer Configuration}
Just like NTP server config, the peer configuration is also handled by \textbf{chronyd} and thus the configuration file is \verb|/etc/chrony.conf|. To set \textit{time.google.com} as the server, and \textit{dns.somuvmnet.local} and \textit{prime.somuvmnet.local} as the peers, the configuration would be:

\vspace{-15pt}
\begin{minted}{lighttpd}
server 	 time.google.com
peer		dns.somuvmnet.local
peer		prime.vm.somuvmnet.local
\end{minted}
\vspace{-10pt}	

\noindent
As with all changes in the configuration of services and daemons, the chronyd service needs to be restarted for the changes to take effect. We do this via:

\vspace{-15pt}
\begin{minted}{console}
# systemctl restart chronyd
\end{minted}
\vspace{-10pt}	

\noindent
Now, chronyc can be used for monitoring, with the command:

\vspace{-15pt}
\begin{minted}{console}
# chronyc sources -v
210 Number of sources = 3

.-- Source mode  '^' = server, '=' = peer, '#' = local clock.
/ .- Source state '*' = current synced, '+' = combined , '-' = not combined,
| /   '?' = unreachable, 'x' = time may be in error, '~' = time too variable.
||                                                 .- xxxx [ yyyy ] +/- zzzz
||      Reachability register (octal) -.           |  xxxx = adjusted offset,
||      Log2(Polling interval) --.      |          |  yyyy = measured offset,
||                                \     |          |  zzzz = estimated error.
||                                 |    |           \
MS Name/IP address         Stratum Poll Reach LastRx Last sample               
===============================================================================
^? fwdns2.vbctv.in               0   6     0     -     +0ns[   +0ns] +/-    0ns
^? 14.139.56.74                  0   6     0     -     +0ns[   +0ns] +/-    0ns
^? 139.59.43.68                  0   6     0     -     +0ns[   +0ns] +/-    0ns
\end{minted}
\vspace{-10pt}	

\noindent
To get further detailed information about the time tracking between the time servers, we can use:

\vspace{-15pt}
\begin{minted}{console}
# chronyc tracking
Reference ID    : 0E8B384A (14.139.56.74)
Stratum         : 3
Ref time (UTC)  : Wed Mar 28 07:14:54 2018
System time     : 0.000516075 seconds slow of NTP time
Last offset     : -0.000552038 seconds
RMS offset      : 0.000552038 seconds
Frequency       : 43.027 ppm slow
Residual freq   : -20.514 ppm
Skew            : 4.453 ppm
Root delay      : 0.057402052 seconds
Root dispersion : 0.044555884 seconds
Update interval : 64.8 seconds
Leap status     : Normal
\end{minted}
\vspace{-10pt}	

