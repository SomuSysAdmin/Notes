\documentclass{report}
\usepackage[utf8]{inputenc}

%	Changing document font to Helvetica.
\usepackage[scaled]{helvet}
\renewcommand\familydefault{\sfdefault} 
\usepackage[T1]{fontenc}

%	Changing Margins and other formatting
\usepackage{geometry}
\geometry{
	a4paper,
	total={170mm,257mm},
	left=1.5in,
	top=1in,
	right=1.5in,
	bottom=1in
}
\setlength{\parskip}{1em}

%	Source Code Highlighting
\usepackage{minted}
%	For Console
\setminted[console]{
frame=lines,
framesep=2mm,
baselinestretch=1.2,
fontsize=\footnotesize,
linenos,
breaklines
}
%	For Shell Scripts
\setminted[bash]{
	frame=lines,
	framesep=2mm,
	baselinestretch=1.2,
	fontsize=\footnotesize,
	linenos,
	breaklines
}
%	For HTML Pages
\setminted[html]{
	frame=lines,
	framesep=2mm,
	baselinestretch=1.2,
	fontsize=\footnotesize,
	linenos,
	breaklines
}
%	For HTTP Config Files
\setminted[lighttpd]{
	frame=lines,
	framesep=2mm,
	baselinestretch=1.2,
	fontsize=\footnotesize,
	linenos,
	breaklines
}
%	For XML Files
\setminted[xml]{
	frame=lines,
	framesep=2mm,
	baselinestretch=1.2,
	fontsize=\footnotesize,
	linenos,
	breaklines
}

%	Pretty Tables
\usepackage{booktabs}
\usepackage{array, multirow}

%	Custom column for tables
\newcolumntype{P}[1]{ >{\centering\arraybackslash} m{#1\linewidth} }
\newcolumntype{M}[1]{m{#1\linewidth}}

%	Images Support
\usepackage{graphicx}

%	Support for spaces in file names
\usepackage[space]{grffile}

%	SUPPORT FOR WEIRD CHARACTERS
\DeclareUnicodeCharacter{25CF}{$\bullet$}
\usepackage{pmboxdraw}

%	Ignore Pygments Bugs & Errors
\AtBeginEnvironment{minted}{%
	\renewcommand{\fcolorbox}[4][]{#4}}

%	Auto-generate Outline
\usepackage[hidelinks]{hyperref}
%	Added packages below just for the sake of autocomplete.
\usepackage{minted}
\usepackage{booktabs}

\begin{document}
	\input{"Mod4/chapters/4.17 Shell Scripting"}		
	
	\section{Using Positional Parameters}
	Anything that's provided to a script as an argument becomes a positional parameter. For example, in the command \verb|ls -l /etc|, the first positional parameter, i.e., \verb|$1| is the value \verb|-l| while the second parameter \verb|$2| is \verb|/etc|. One \textbf{wrong} way to use positional parameters in a script would be:
	
	\vspace{-15pt}
	\begin{minted}{bash}
	#!/bin/bash
	
	echo parameter 1: $1
	echo parameter 2: $2
	echo parameter 3: $3
	\end{minted}
	\vspace{-10pt}	
	
	\noindent
	This script works as expected when 3 positional arguments are provided: 
	
	\vspace{-15pt}
	\begin{minted}{console}
	$ ./ex3 a b c 
	parameter 1: a
	parameter 2: b
	parameter 3: c
	$ ./ex3 a b c d e f
	parameter 1: a
	parameter 2: b
	parameter 3: c
	\end{minted}
	\vspace{-10pt}	
	
	\noindent
	In the second case, when more than 3 arguments are given, the ones after the $3^{rd}$ argument are simply ignored. But when the number of arguments is just 1, the second and the third line are executed anyway:
	
	\vspace{-15pt}
	\begin{minted}{console}
	$ ./ex3 a 
	parameter 1: a
	parameter 2:
	parameter 3:
	\end{minted}
	\vspace{-10pt}	
	
	\noindent
	We should intelligently find out the number of arguments provided to a script and then treat them accordingly. For this, we need a mechanism like a for loop, which iterates over the contents of an item. So, the script becomes:
	
	\vspace{-15pt}
	\begin{minted}{bash}
	#!/bin/bash
	count=1
	for i in "$@";
	do
		echo "parameter $count : $i"
		((count++))
	done
	\end{minted}
	\vspace{-10pt}	
	
	\noindent
	The count variable is used to keep track of the position of the argument being displayed, while the \verb|$@| variable is actually an array (i.e., a variable that stores multiple values of the same type) which contains all of the positional arguments. The final line, \verb|((count++))| increments the value of count and is called an \textit{arithmetic expression} in bash. Such expressions must always occur inside the double brackets to be evaluated. The output of this script is:
	
	\vspace{-15pt}
	\begin{minted}{console}
	$ ./ex4 a b c d e f
	parameter 1 : a
	parameter 2 : b
	parameter 3 : c
	parameter 4 : d
	parameter 5 : e
	parameter 6 : f
	$ ./ex4 a 
	parameter 1 : a
	$ ./ex4
	$
	\end{minted}
	\vspace{-10pt}	
	
	\noindent
	
	
	\section{Understanding if then else}
	\section{Understanding for}
	\section{Understanding while and until}
	\section{Understanding case}
		
\end{document}