\chapter{Managing Advanced Apache Features}

\section{Setting up Authenticated Web Servers}
\subsection{The HTTPD Manual}
The first thing that we're going to do is use \verb|yum -y install httpd-manual| to install the httpd-manual. This package installs a set of local web-pages that serve as a manual for all Apache configurations. Thus, it provides an easy way to look up information and/or commands offline when needed. Once installed, the manual can be browsed from \verb|http://localhost/manual|.

\subsection{Apache Users for basic Authentication}
Let us consider a website with 3 sections: a public space, a private space and an exclusive space for a certain user '\textit{lisa}'. For authentication we need users - and rather than use Linux users, Apache has it's own system to create users. The users and their passwords are stored in password-files, which are created using the \verb|htpasswd| utility that comes bundled with Apache. To create a password and show it on screen, we use \verb|htpasswd -n <username>| and enter the password when prompted:

\vspace{-15pt}
\begin{minted}{console}
# htpasswd -n somu
New password: 
Re-type new password: 
somu:$apr1$ejEfs5E7$wzlrLgYWYNKSrG7BQVLIa1
\end{minted}
\vspace{-10pt}	

\noindent
However, this is of no use for authentication, since the information isn't stored and Apache can't use it to verify users. Thus, to securely store the password in some passwordfile (that is inaccessible from the internet, so that people can't just \textit{download} it), we choose to store it in \verb|/etc/httpd/htpasswd| with:

\vspace{-15pt}
\begin{minted}{console}
# htpasswd -c /etc/httpd/htpasswd lisa
New password: 
Re-type new password: 
Adding password for user lisa
# cat /etc/httpd/htpasswd
lisa:$apr1$pUh9Uxin$zCRJuoWcbkkpDBw04ZaxS0
\end{minted}
\vspace{-10pt}	

\noindent
The \verb|-c| option enables the creating of a new password file, or replace (overwrite) an existing one if present. It's very important to only use the -c option for a brand new password file. Every subsequent use should be without any option:

\vspace{-15pt}
\begin{minted}{console}
# htpasswd /etc/httpd/htpasswd lori
New password: 
Re-type new password: 
Adding password for user lori
# cat /etc/httpd/htpasswd
lisa:$apr1$r6Xj/zbR$MTFl/9Oq/vcmO1.PLue5W0
lori:$apr1$gZ8ZnGGD$q3GL3wpB0T.JTCa2pw/jD0
\end{minted}
\vspace{-15pt}	

\subsection{Directory rules in httpd.conf}
Now that we have users, we need to add the specifications for the protected directory in the Apache configs, and dictate when the server should ask for a password and which users should be allowed access. The next part of the required config can either be added in \verb|/etc/httpd/conf/httpd.conf| (preferably at the very bottom to maintain organization) or in a seperate \verb|.htaccess| file within the directory whose access it'll control (For this, the \verb|AllowOverride| directive for the directory can't be set to \verb|none|). In either case, the \verb|<Directory>| directives must be defined in the \verb|httpd.conf| file, or any \verb|.conf| file in the \verb|conf.d| directory. Let us assume we've set up a virtual host called \textit{authtest.somuvmnet.local} which will house the files we need. So, the file \verb|/etc/httpd/conf.d/authtest.conf| will contain:

\vspace{-15pt}
\begin{minted}{lighttpd}
# Rules for the directory private and all its subdirs (which have .htaccess files)
<Directory "/var/www/html/authtest/private">
	AllowOverride all
</Directory>
# Virtual host config
<VirtualHost *:80>
	DocumentRoot /var/www/html/authtest
	ServerName 	authtest.somuvmnet.local
	ServerAlias 	 aut.somuvmnet.local
	ServerAlias 	 aut.svmn.loc
	ServerAdmin	root@aut.somuvmnet.local
	ErrorLog 	"logs/aut_error_log"
	CustomLog	"logs/aut_access_log" combined
</VirtualHost>
\end{minted}
\vspace{-15pt}	

\noindent
In the next step, either the steps in Section 9.1.4 or 9.1.5 should be followed:

\vspace{-10pt}
\subsection{Controlling access via .htaccess files}
The final directory structure of our site will look like:

\vspace{-15pt}
\begin{minted}{console}
# tree -a /var/www/html/authtest
/var/www/html/authtest
├── index.html
└── private
	├── .htaccess
	├── index.html
	└── lisaZone
		├── .htaccess
		└── index.html
\end{minted}
\vspace{-10pt}	

\noindent
For example, if we're trying to restrict access to \verb|/var/www/html/authtest/private| we'll add a new \verb|.htaccess| file in it, with the contents: 

\vspace{-15pt}
\begin{minted}{lighttpd}
AuthType Basic
AuthName "Private Space"
AuthUserFile /etc/httpd/htpasswd        
Require valid-user
\end{minted}
\vspace{-10pt}	

\noindent
Now, we create a directory \verb|/var/www/html/authtest/private/lisaZone| and create a \verb|.htaccess| file within it with the contents:

\vspace{-15pt}
\begin{minted}{lighttpd}
# Only one user (lisa) allowed in lisaZone
AuthType Basic
AuthName "lisaZone"
AuthUserFile /etc/httpd/htpasswd        
Require user lisa
\end{minted}
\vspace{-10pt}	

\noindent
The above \verb|.htaccess| files sets up the permissions for two folders: \textit{private} and \textit{private/lisaZone}. While any valid Apache user is allowed in the \verb|private| directory, only user lisa can enter \verb|private/lisaZone|, due to the \verb|Require user| directive. 

\subsection{Controlling access via httpd.conf}
If we were to put these settings in \verb|httpd.conf| instead of \verb|.htaccess| files in the proper directories, we'd need \verb|<Directory>| directives to define the location where these'd be applied. The following lines would then need to be added to \verb|httpd.conf|:

\vspace{-15pt}
\begin{minted}{lighttpd}
<Directory /var/www/html/authtest/private>
	AllowOverride none
	AuthType Basic
	AuthName "Private Space"
	AuthUserFile /etc/httpd/htpasswd        
	Require valid-user
</Directory>

<Directory /var/www/html/authtest/private/lisaZone>
	AllowOverride none
	AuthType Basic
	AuthName "lisaZone"
	AuthUserFile /etc/httpd/htpasswd        
	Require user lisa
</Directory>
\end{minted}
\vspace{-10pt}	

\noindent
Finally, we add the html content for the site.

\subsection{Adding HTML Content}
The \verb|/var/www/html/authtest/index.html| should contain:

\vspace{-15pt}
\begin{minted}{html}
<html>
	<head>
		<title>Public Space</title>
	</head>
	<body>
		<h1>Welcome to the public space!</h1>
		<p>This portion of the website should be accessible by all!</p>
		<p>The only reason this page exists is to act as a control page to test the reactions of the private page to the new configs. Regardless, try to click the link below and see if you can actually access it...</p>
		<a href="private/">Click me to Enter the Restricted Area!</a>
	</body>
</html>
\end{minted}
\vspace{-10pt}	

\noindent
The \verb|/var/www/html/authtest/private/index.html| should contain:

\vspace{-15pt}
\begin{minted}{html}
<html>he \verb|/var/www/html/authtest/private/index.html| should contain:
	<head>
		<title>Private Space</title>
	</head>
	<body>
		<h1>Welcome to the PRIVATE Space</h1>
		<p>This portion of the website should be accessible to only authenticated users</p>
		<p>If you can see this page without authenticating something is wrong with the configs!!!</p>
		<a href="../">Go Back</a>
		<a href="lisaZone/">Go to lisaZone!</a>
	</body>
</html>
\end{minted}
\vspace{-10pt}	

\noindent
Finally, the \verb|/var/www/html/authtest/private/lisaZone/index.html| should contain:

\vspace{-15pt}
\begin{minted}{html}
<html>
	<head>
		<title>LisaZone -- the ultimate protected space</title>
	</head>
	<body>
		<h1>Welcome to the lisaZone</h1>
		<p>This portion of the website should be accessible to user lisa</p>
		<p>If you can see this page and aren't user LISA, something is wrong with the configs!!!</p>
		<p><a href="../">Go Back to Private Zone</a></p>
		<p><a href="../../">Go Back to Public Zone</a></p>
	</body>
</html>
\end{minted}
\vspace{-10pt}	

\noindent
Now our site is ready. So, we need to check the httpd config syntax using \verb|httpd -t|. If the syntax is correct, we restart httpd using \verb|systemctl restart httpd|. Then we visit the website by \verb|elinks http://authtest.somuvmnet.local|. Based on authentication, we should be allowed to access the different parts of the sites. 

\section{Configuring Apache for LDAP Authentication}
Manual user creation via \verb|htpasswd| gets cumbersome and inefficient when large sites are concerned. In those cases, we can choose LDAP authentication instead. This, however, doesn't mean that local users and groups have to be omitted from the config. 

Let us consider an organization \verb|myorg|. Let it contain a group \verb|mygroup|. We want to restrict access to directory \verb|/www/docs/private| to either Apache or LDAP users and groups. Then, configuration in \verb|/etc/httpd/conf/httpd.conf| will look like:

\vspace{-15pt}
\begin{minted}{lighttpd}
<Directory /www/docs/private>
	AuthType Basic
	AuthName "Private"
	AuthBasicProvider file
	AuthUserFile /usr/local/apache/passwd/passwords
	AuthLDAPURL ldap://ldaphost/o=myorg
	AuthGroupFile /usr/local/apache/passwd/groups
	Requrie group GroupName
	Require ldap-group cn=mygroup,o=myorg
</Directory>
\end{minted}
\vspace{-10pt}	

\noindent
Here, we have an order of checking for the users/groups, just like in the case of the Pluggable Authentication Module (PAM). First Apache tries to find the user in the password file. If it can't \textit{then} it checks LDAP.

	\section{Enabling CGI Scripts}
On any web server, there might be a requirement for dynamic content. This kind of content is generated by a script using a server side scripting language such as CGI, PHP or even python. Scripts becomes crucial when databases are involved since the scripts often fetch information from a database. 

\subsection{CGI}
CGI is an abbreviation for \textbf{Common Gateway Interface}. It is a specification for information transfer between a web server (such as Apache) and a CGI program/script. CGI is the oldest standard, and even though it can be used by both PHP and python, it's not optimal. To use CGI, we need to use:

\vspace{-15pt}
\begin{minted}{lighttpd}
ScriptAlias	/cgi-bin/	"/var/www/cgi-bin"
\end{minted}
\vspace{-10pt}	

\noindent
The CGI scripts must be executable by the \verb|apache| user and group. There are also a certain file context (\verb|httpd_sys_script_exec_t|) on the directory \verb|/var/www/cgi-bin| which is needed by SELinux to permit the execution of such scripts. 

\subsection{PHP}
\textbf{PHP} has been much more common than CGI for quite some time now. For PHP scripting to be enabled, the \textbf{mod\_php} Apache module must be installed and loaded. This simple act itself adds all the necessary configuration to the http configuration, such as setting:

\vspace{-15pt}
\begin{minted}{lighttpd}
SetHandler	application/x-httpd-php
\end{minted}
\vspace{-10pt}	

\noindent
This line ensures that PHP can be run from the Apache web server, and other than the installation of \textit{mod\_php} Apache module, no manual intervention is needed by default. 

\subsection{Python}
In case of python, the name of the required module is \textbf{mod\_wsgi}. Then we'd need to define a \verb|WSGIScriptAlias| to redirect to the correct application:

\vspace{-15pt}
\begin{minted}{lighttpd}
WSGIScriptAlias	/myapp/	/srv/myapp/www/myapp.py
\end{minted}
\vspace{-10pt}	

\noindent
To connect to a local database, no additional configuration besides that in the script is necessary. However, if the database is a remote database, then certain SELinux booleans need to be set to true. These are: \verb|httpd_can_network_connect_db| and depending on the database we're using, perhaps \verb|httpd_can_network_connect| as well.

	\section{Understanding TLS protected Web Sites} 
TLS stands for \textbf{Transport Layer Security} and it performs data encryption and identity verification to enhance security. For example, if we visit the website of our bank, we would want to ensure that it's indeed the website of the bank that we're visiting and not some other site that some nefarious agent may have copied to steal data. 

Further, we'd also want to ensure that the login credentials, or our personal data that the bank holds (such as account numbers or balances) are not being leaked during transit. Both of these features are provided by TLS. The entire basis of TLS are certificates that act as public keys for websites. 

\subsection{Certificate Authorities}
The validity of the certificates are guaranteed by a Certificate Authority (CA), who are 3rd party agents who verify that the organization handing out the certificate are the true owner of the server you're about to access. If so, they sign a digital certificate and provide it to them which they can then give to people who're interested in visiting their site. Now, when we communicate with the site, we can by ourselves check whether their credentials match the one on the certificate to determine if we're communicating with the right server. 

\subsection{Self-Signed Certificates}
Certificates can be self-signed too. There are mechanisms via which any server can generate it's own certificate and provide/install it on a client's computer. However, we can't be sure if the organization who just provided us the certificate is really who they claim to be. For example, a site impersonating our bank's website may also hand out a TLS Certificate that matches it's signature. Now, unless we involve CAs, there's no way for us to determine which certificate is the genuine one belonging to our bank. 

However, for testing environments, a self-signed certificate is good enough, since no actual valuable data is being passed around, and in case of internal networks, attacks such as \textit{man-in-the-middle} attack (which TLS Certs actively protect against) are useless/impractical. However, signed certificates are essential for production due to the concerns noted above. 

\subsection{Asymmetrical Encryption}
\textbf{NOTE}: This particular section is \textbf{optional} since it deals with a bit of complicated mathematics, and is irrelevant to the RHCE course, and should only be interesting to those who want a glimpse at the inner mechanisms of public-private key encryption. 

Cryptography is a branch of mathematics that deals in creating algorithms of functions that can scramble information. Much like a padlock that needs a key, a number (or a set of them) is required to scramble and unscramble this message. Thus, we refer to such a number as a key. The entire basis of cryptography is contingent upon creating messages that are hard to decode without a certain key. As such, there are two possible methods: Symmetric encryption and Asymmetric encryption. 

\subsubsection{Method of Symmetric Encryption}
\vspace{-10pt}
In the case of symmetric encryption, there is a function \textit{Encrypt} $E_k(m)=f(m,k)$ that takes a message $m$ and encrypts it with a key $k$. Let this output be a \textit{cypher-text} denote by $E_k(m)=f(m,k)=c$. Now, if this same key can be used to decrypt the message with a function Decrypt ($D_k(c)=f^{-1}(c,k)$) that does the opposite actions of function $f$, i.e. $D_k(c)=f^{-1}(c,k)=m$,  then we've got symmetric cryptography. So, it the rules for symmetric crypto are: 

\begin{equation}
	E_k(m) = f(m,k) = c
\end{equation}
\vspace{-30pt}

\begin{equation}
	D_k(c) = f^{-1}(c,k) = m
\end{equation}

\noindent
So, the initiator of the communication simply performs $E_k(m)$ to get the encrypted message $e$ and sends it to the recipient. The recipient, who is also in possession of the key, then performs $D_k(c)$ to obtain the original message $m$. 

The drawback of this form of crypto is that the key must be known to both parties. However, this means that the key itself has to be shared over some means of encryption or risk some 3rd party intercepting the key and gaining access to privileged communication. This problem is so serious that spies in the cold war used to meet up in secret and share encryption/decryption keys in envelopes. Thankfully, we now have a better way.

\subsubsection{Method of Asymmetric Encryption}
\vspace{-10pt}
In the case of asymmetric encryption we need two keys, instead of just one: we'll called them a public key ($k_p$) and private (secret) key ($k_s$). The encryption function $E_{k_p}(m)=f_1(m,k_p)=c$ now uses the public key ($k_p$) to encrypt the message, and the decryption function $D_{k_s}(c)=f_2(c,k_s)=m$ uses the private (secret) key ($k_s$) to decrypt it. So, we have:

\begin{equation}
	E_{k_p}(m)=f_1(m,k_p)=c
\end{equation}
\vspace{-30pt}

\begin{equation}
	D_{k_s}(c)=f_2(c,k_s)=m
\end{equation}

\noindent
The functions $E_{k_p}$ and $D_{k_s}$ are often \textit{one-way} functions which are a special category of functions that produce an output so complex, that the computational power and/or time required to reverse engineer the functions from the given input and output without the key is near zero. Contrastingly, a one-way function is easy to solve: for function $s(i)=i^2$, we get $s(2)=2^2=4$ and we can reverse engineer it to find $s^{-1}(4)=\sqrt{4}=2$, if we know the input and output, i.e., $2$ and $4$. 

\subsection{Mathematics of Asymmetrical Encryption}
\subsubsection{Encryption step}
\vspace{-10pt}
Practically in RSA encryption, the keys are the product of two \textbf{gargantuan} prime numbers, while the cypher-text is based on modular arithmetic. Thus, it becomes computationally intensive enough to be declared \textit{practically} impossible to factor the key to generate the original set of prime numbers. An example would be where the encryption (private/secret) key $N$ is the product of two primes $p_1$ and $p_2$. We'd also choose another number $e$ such that the result of $((p_1-1) \times (p_2-1))$ and $e$ are relatively prime, i.e., they don't share a common factor. Then, the encryption function for a message $m$ is:

\begin{align}
	E_{k_s}(m) &= m^e(\mod N) \\
	&= m^e(\mod (p_1 \times p_2)) \text{, where:} \\
	N &= (p_1 \times p_2) \\
	1 &= \operatorname{GCD}(e,(p_1-1) \times (p_2-1))
\end{align}
\vspace{-35pt} 
\begin{center}
	i.e., $p_1$ and $p_2$ are relatively prime (when greatest common divisor=1)
\end{center}

\subsubsection{Decryption step}
\vspace{-10pt}
In this step, we need yet another number which is the decryption key $d$ such that:

\begin{align}
	e \times d &\cong 1\mod ((p_1 -1) \times (p_2 - 1)) \text{, i.e.,} \\
	1 &= ed \mod (p_1-1)(p_2-1)
\end{align}

\noindent
Due to the above relation, it's so important that N and e be relatively prime! (otherwise, due to the nature of congruity, and due to the many-to-1 mapping of the modulus function's input and output, weird things happen). And the decrypted text, i.e., the original message $m$ is given by:

\begin{align}
	D_{k_p}(c) &= c^d(\mod N) \\
	&=c^d (\mod (p_1 \times p_2)) \\ 
	&= m 
\end{align}

\noindent
So, the only way to know the decryption number $d$ is to know the numbers $p_1$ and $p_2$. Then our \textbf{private key} is the set of numbers $p_1$ and $p_2$ while the public keys are the set of the numbers $N$ and $e$:

\begin{align}
	\text{Private Key} : k_s &= \{ p_1, p_2 \} \\
	\text{Public Key} : k_p &= \{ N, e \}
\end{align}

Since alphanumeric characters can be mapped to their decimal equivalent via ASCII/UTF-8/etc conversion standards, they can be encrypted and decrypted via asymmetrical encryption called public key encryption. 

\subsection{Implications of the Encryption}
Due to the nature of these keys, only a message encrypted by a private key can only be decrypted by the proper public key. Thus, utilities like the \verb|ssh-keygen| and \verb|genkey| always generate key-pairs, i.e, both private and public key together. These keys can be used in two ways: 

\vspace{-10pt}
\begin{itemize}
	\item Encryption for transmission
	\item Authentication
\end{itemize}
\vspace{-10pt}

\noindent
Since only the message encrypted with the private key can be transformed to form the original message when decrypted with the public key, a person or an organization could convert a pre-determined message (that is expected by the recipient) to cypher-text using their private key. Now, recipient simply decrypts the message with the public key of sender. If the message is as expected, then we know that only the sender could've encrypted it with the private key, since no one else has it, thus verifying his/her/their identity. Both of these properties are used by TLS certificates. 

\section{Setting up TLS protected Web Sites}
\subsection{Generating the TLS Certificate}
To generate a TLS certificate, we need the \textbf{crypto-utils} package and the \textbf{mod\_ssl} Apache plugin. So, we execute:

\vspace{-15pt}
\begin{minted}{console}
# yum -y install crypto-utils mod_ssl
\end{minted}
\vspace{-10pt}	

\noindent
Within the crypto-utils package an utility called \verb|genkey| is capable of generating TLS certificates. We launch the TUI for it using:

\vspace{-15pt}
\begin{minted}{console}
# genkey vmPrime.somuvmnet.local
\end{minted}
\vspace{-10pt}	

\noindent
The utility then notifies us that a \textbf{key} will be stored in \verb|/etc/pki/tls/private/| location with the file name \verb|vmPrime.somuVMnet.local.key|. This is our private key. The public key is the certificate that we give everyone, and is stored in \verb|/etc/pki/tls/certs/| with the name \verb|vmPrime.somuVMnet.local.crt|. Thus the important things to remember are:

\noindent
\begin{tabular}{rlll}
	\toprule
	\textbf{Key type} &\textbf{Location} &\textbf{Name}\\
	\midrule
	\textbf{Private Key}	&\verb|/etc/pki/tls/private/|	&vmPrime.somuVMnet.local.key\\
	\textbf{Public Key}	&\verb|/etc/pki/tls/certs/|	&vmPrime.somuVMnet.local.crt\\
	\midrule
	\textit{Common Path} &\verb|/etc/pki/tls/| &\\
	\textit{Naming Convention} &\verb|<FQDN>.key| \textbf{OR} \verb|<FQDN>.crt| &\\
	\bottomrule
\end{tabular}

\noindent
Next, we're asked about the security level used in the certificates, and the \textit{2048 bits} used are good enough for most purposes. The shorter the key, the faster the response time for the server, but the key is also easier to crack. 

Once the key generation is completed, it asks us if we'd like the certificate to be sent to a CA for it to be a signed. Since this is going to be a self-signed certificate, we choose No. Next, we're asked if we'd like to encrypt the private key using a pass-phrase. If we choose to encrypt it, we'll have to enter the pass-phrase (that should be the same for all keys installed on a server installation) every time the web server boots. If left unencrypted, anyone who steals the file from the server can decrypt the communication. Encrypting the private key makes this significantly harder. This should be followed for any real installation. However, for our test, we'll choose not to encrypt. 

Next, we're asked for information about the server needed to form its \textit{Distinguished Name (DN)}. While the rest of the information is used for non-technical purposes, the common name is the most important part. The common name must match the server name! Upon choosing "next" the certificate will be generated:

\vspace{-15pt}
\begin{minted}{console}
# genkey vmPrime.somuVMnet.local
/usr/bin/keyutil -c makecert -g 2048 -s "CN=vmPrime.somuVMnet.local, O=SomuVMnet, L=Bangalore, ST=Karnataka, C=IN" -v 1 -a -z /etc/pki/tls/.rand.2953 -o /etc/pki/tls/certs/vmPrime.somuVMnet.local.crt -k /etc/pki/tls/private/vmPrime.somuVMnet.local.key
cmdstr: makecert

cmd_CreateNewCert
command:  makecert
keysize = 2048 bits
subject = CN=vmPrime.somuVMnet.local, O=SomuVMnet, L=Bangalore, ST=Karnataka, C=IN
valid for 1 months
random seed from /etc/pki/tls/.rand.2953
output will be written to /etc/pki/tls/certs/vmPrime.somuVMnet.local.crt
output key written to /etc/pki/tls/private/vmPrime.somuVMnet.local.key


Generating key. This may take a few moments...

Made a key
Opened tmprequest for writing
/usr/bin/keyutil Copying the cert pointer
Created a certificate
Wrote 1682 bytes of encoded data to /etc/pki/tls/private/vmPrime.somuVMnet.local.key 
Wrote the key to:
/etc/pki/tls/private/vmPrime.somuVMnet.local.key
\end{minted}
\vspace{-10pt}	

\noindent
We can see that the certificates have been generated with the correct SELinux context label of \verb|cert_t|:

\vspace{-15pt}
\begin{minted}{console}
# ls -Z /etc/pki/tls/private/vmPrime.somuVMnet.local.key; ls -Z /etc/pki/tls/certs/vmPrime.somuVMnet.local.crt 
-r--------. root root unconfined_u:object_r:cert_t:s0  /etc/pki/tls/private/vmPrime.somuVMnet.local.key
-rw-r-----. root root unconfined_u:object_r:cert_t:s0  /etc/pki/tls/certs/vmPrime.somuVMnet.local.crt
\end{minted}
\vspace{-10pt}	

\subsection{Setting up Apache to use the TLS Certificates}
The installation of \verb|mod_ssl| creates a configuration file called \verb|/etc/httpd/conf.d/ssl.conf|. The important parameters in this file are:

\vspace{-15pt}
\begin{minted}{lighttpd}
Listen 443 https
\end{minted}
\vspace{-10pt}	

\noindent
This instructs the httpd process to listen for incoming https connections on port \verb|443|, which is the designated port for HTTPS by default. In this default configuration, the server listens to both port 80 (HTTP) and 443 (HTTPS), but if we really want our server to be secure, we only want to accept incoming HTTPS connections, and thus we should ask the httpd process to refrain from listening on port 80 and the port 80 should be closed. We could setup a virtual host and have it listen only on port 443. 

By default this configuration is set to use a file called \verb|localhost.crt| and \verb|localhost.key|. However, we must use our self-signed certificates here, by changing the paths to the correct file paths:

\vspace{-15pt}
\begin{minted}{lighttpd}
SSLCertificateFile /etc/pki/tls/certs/vmPrime.somuVMnet.local.crt
SSLCertificateKeyFile /etc/pki/tls/private/vmPrime.somuVMnet.local.key
\end{minted}
\vspace{-10pt}	

\noindent
Now, we can close the file, check the syntax with \verb|httpd -t| and if all is ok, we restart the server using:

\vspace{-15pt}
\begin{minted}{console}
# systemctl restart httpd
\end{minted}
\vspace{-10pt}	

\noindent
Now when we try to visit the site using firefox with the command: \verb|firefox https://localhost|, firefox is going to warn about the fact that the certificate is self-signed, and the fact that localhost doesn't match the CN of the certificate (vmPrime.somuVMnet.local). We can add an exception and move on. 

\subsection{Configuring TLS Certs for Virtual Hosts}
While everything else remains the same, the \verb|genkey| utility will need to be invoked with the name of the virtual host itself. Then, on the configuration for the virtual host, the following lines should be added:

\vspace{-15pt}
\begin{minted}{lighttpd}
<VirtualHost *:443>
	ServerName	sales.example.com
	DocumentRoot	/web/sales
	SSLEngine	on
	SSLCertificateFile	/etc/pki/tls/certs/sales.example.com.crt	
	SSLCertificateKeyFile	/etc/pki/tls/private/sales.example.com.key	
</VirtualHost>
\end{minted}
\vspace{-10pt}	