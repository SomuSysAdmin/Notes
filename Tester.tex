\documentclass{report}
\usepackage[utf8]{inputenc}

%	Changing document font to Helvetica.
\usepackage[scaled]{helvet}
\renewcommand\familydefault{\sfdefault} 
\usepackage[T1]{fontenc}

%	Changing Margins and other formatting
\usepackage{geometry}
\geometry{
	a4paper,
	total={170mm,257mm},
	left=1.5in,
	top=1in,
	right=1.5in,
	bottom=1in
}
\setlength{\parskip}{1em}

%	Source Code Highlighting
\usepackage{minted}
%	For Console
\setminted[console]{
frame=lines,
framesep=2mm,
baselinestretch=1.2,
fontsize=\footnotesize,
linenos,
breaklines
}
%	For Shell Scripts
\setminted[bash]{
	frame=lines,
	framesep=2mm,
	baselinestretch=1.2,
	fontsize=\footnotesize,
	linenos,
	breaklines
}
%	For HTML Pages
\setminted[html]{
	frame=lines,
	framesep=2mm,
	baselinestretch=1.2,
	fontsize=\footnotesize,
	linenos,
	breaklines
}
%	For HTTP Config Files
\setminted[lighttpd]{
	frame=lines,
	framesep=2mm,
	baselinestretch=1.2,
	fontsize=\footnotesize,
	linenos,
	breaklines
}
%	For XML Files
\setminted[xml]{
	frame=lines,
	framesep=2mm,
	baselinestretch=1.2,
	fontsize=\footnotesize,
	linenos,
	breaklines
}

%	Pretty Tables
\usepackage{booktabs}
\usepackage{array, multirow}

%	Custom column for tables
\newcolumntype{P}[1]{ >{\centering\arraybackslash} m{#1\linewidth} }
\newcolumntype{M}[1]{m{#1\linewidth}}

%	Images Support
\usepackage{graphicx}

%	Support for spaces in file names
\usepackage[space]{grffile}

%	SUPPORT FOR WEIRD CHARACTERS
\DeclareUnicodeCharacter{25CF}{$\bullet$}
\usepackage{pmboxdraw}

%	Ignore Pygments Bugs & Errors
\AtBeginEnvironment{minted}{%
	\renewcommand{\fcolorbox}[4][]{#4}}

%	Auto-generate Outline
\usepackage[hidelinks]{hyperref}
%	Added packages below just for the sake of autocomplete.
\usepackage{minted}
\usepackage{booktabs}

\begin{document}
	\input{"RHCE/Mod1/chapters/1.2 Configuring iSCSI Target and Initiator"}
	
	\subsection{Adding a rule to the firewall}
	Now, we need to allow the TCP connections through port 3260 to use for SAN, using:
	
	\vspace{-15pt}
	\begin{minted}{console}
	# firewall-cmd --add-port=3260/tcp --permanent 
	success
	# firewall-cmd --reload
	success
	\end{minted}
	\vspace{-10pt}
	
	\subsection{Starting target.service}
	Even though \textbf{targetcli} saves the present configuration to disk, a service called \textit{target.service} must be enabled to ensure that the saved configuration is loaded each time after reboots. This is done with:
	
	\vspace{-15pt}
	\begin{minted}{console}
	# systemctl start target
	# systemctl enable target
	Created symlink from /etc/systemd/system/multi-user.target.wants/target.service to /usr/lib/systemd/system/target.service.
	# systemctl status target
	● target.service - Restore LIO kernel target configuration
	Loaded: loaded (/usr/lib/systemd/system/target.service; enabled; vendor preset: disabled)
	Active: active (exited) since Tue 2018-01-02 16:28:20 IST; 25s ago
	Main PID: 4291 (code=exited, status=0/SUCCESS)
	
	Jan 02 16:28:19 vmprime.somuvmnet.local systemd[1]: Starting Restore LIO kern...
	Jan 02 16:28:20 vmprime.somuvmnet.local systemd[1]: Started Restore LIO kerne...
	\end{minted}
	\vspace{-10pt}
	
	\noindent
	
	
	\section{Connecting the iSCSI Initiator to an iSCSI SAN}
	\section{Verifying the iSCSI Connection}	
	
\end{document}


