\chapter{IPv4 Addressing}
Even though the global pool of IPv4 addresses has been exhausted, and IPv6 is the future, we still have a lot of IPv4 addressing in networking today. Thus, we need the following basic skills: 
\vspace{-10pt}
\begin{itemize}
	\item Calculating the number of subnets for a subnet mask.
	\item Calculating the number of hosts for a subnet mask.
	\item Calculating the usable address range for a specific subnet.
	\item Finding the Broadcast ID, Host IDs and Network ID in a network, etc. 
\end{itemize}

\section{Binary Numbering}
An example of an IPv4 Address (or simply, an IP address) is \textbf{10.1.2.3}. Before we can start working with IP addresses and use them to design networks, and calculate the specifics of a network, we need to be able to convert an IP address into its binary equivalent. An IPv4 address is 32-bits, and we use the \textbf{Dotted Decimal notation} to represent it, dividing the 32 bits into four sets of 8 bits each, called an \textbf{octet} and then writing the decimal equivalent of each octet, separated by dots, to represent the IP address. 

\begin{center}
	\begin{tabular}{rcccc}
		\toprule
		\textbf{Dotted Decimal Notation} &10 &1 &2 &3 \\
		\midrule
		\textbf{Binary Equivalent}	&1010 &1 &10 &11\\
		\textbf{8-bit Binary (Octet)}	&00001010 &00000001 &00000010 &00000011\\
		\textbf{Octet Number} &$1^{st}$ Octet &$2^{nd}$ Octet &$3^{rd}$ Octet &$4^{th}$ Octet  \\
		\bottomrule
	\end{tabular}
\end{center}

\subsection{Binary to Decimal Conversion}
Since binary numbers are \textbf{base-2} like decimal numbers are \textbf{base-10}, we need a binary conversion table such as the one shown below to convert a binary number to it's decimal equivalent.

\begin{table}[H]
	\centering
	\begin{tabular}{cccccccc}
		\toprule
		$2^7$ &$2^6$ &$2^5$ &$2^4$ &$2^3$ &$2^2$ &$2^1$ &$2^0$ \\
		\textbf{128} &\textbf{64} &\textbf{32} &\textbf{16} &\textbf{8} &\textbf{4} &\textbf{2} &\textbf{1} \\
		\bottomrule
	\end{tabular}
	\vspace{-5pt}
	\caption{Binary Conversion Table}
\end{table}
\vspace{-10pt}

\noindent
The above table simplifies the formula: 
\begin{equation}
	\text{Decimal} D = \sum_{i=n}^{0} 2^n b_i 
	\label{eqn:bin2dec}
\end{equation}
\noindent where $b_i$ is the binary bit in the $i^{th}$ position of the binary number $B$, $n$ is the number of digits in $B$ and $D$ is the decimal representation of $B$. 

\subsubsection{Converting 10010110 to decimal}
First we construct the binary conversion table and write the binary digits beneath the powers of 2. 

\begin{table}[H]
	\centering
	\begin{tabular}{cccccccc}
		\toprule
		$2^7$ &$2^6$ &$2^5$ &$2^4$ &$2^3$ &$2^2$ &$2^1$ &$2^0$ \\
		\textbf{128} &\textbf{64} &\textbf{32} &\textbf{16} &\textbf{8} &\textbf{4} &\textbf{2} &\textbf{1} \\
		\midrule
		1 &0 &0 &1 &0 &1 &1 &0 \\
		\bottomrule
	\end{tabular}
	\vspace{-5pt}
	\caption{Binary Conversion Table}
\end{table}
\vspace{-10pt}

\noindent
Now, from the above table, we simply add the powers of 2 (column headings) that have a 1 below them, and ignore those that have a zero in the binary representation to get the decimal equivalent.
\begin{align*}
	d &= 128 + 16 + 4 + 2\\
	  &= 150
\end{align*}
So, the binary number $(10010110)_2$ in decimal is $(150)_{10}$. This can be directly calculated with the equation \ref{eqn:bin2dec}: 
\begin{align*}
	d &= \sum_{i=7}^{0} 2^i b_i \\
	&= (2^7\times1) + (2^6\times0) + (2^5\times0) + (2^4\times1) + (2^3\times0) + (2^2\times1) +( 2^1\times1) + (2^0\times0) \\
	&= 128 + 16 + 4 + 2\\
	&= 150
\end{align*}

\subsection{Decimal to Binary Conversion}
Decimal numbers can also be converted to Binary by using the binary conversion table. In this method, for each column in the binary conversion table, we ask if the number is greater than or equal to the value in the column. If yes, then we put a \textbf{1} in the cell below it and subtract the value of the column from the number, and then repeat the same process with the new value. 

Note that if at any point during the calculation, the remainder turns \textbf{0}, then all the columns to the right will also be \textbf{0}, since the number has been broken down completely into factors of 2. 

\subsubsection{Converting 167 to Binary}
First we draw the binary conversion table. The first column is the value \verb|128|. Since $167>128$, we write \textbf{1} in the cell below 128, and then subtract $167-128=39$. This is the new value we'll compare against. 

\begin{table}[H]
	\centering
	\begin{tabular}{rcccccccc}
		\toprule
		&$2^7$ &$2^6$ &$2^5$ &$2^4$ &$2^3$ &$2^2$ &$2^1$ &$2^0$ \\
		&\textbf{128} &\textbf{64} &\textbf{32} &\textbf{16} &\textbf{8} &\textbf{4} &\textbf{2} &\textbf{1} \\
		\midrule
		\textbf{Quotient} &1 &0 &1 &0 &0 &1 &1 &1 \\
		\midrule
		\textbf{Remainder} &39 &39 &7 &7 &7 &3 &1 &0 \\
		\bottomrule
	\end{tabular}
	\vspace{-5pt}
	\caption{Binary Conversion Table}
\end{table}
\vspace{-10pt}

\noindent
Since $32<64$, we just write a \textbf{0} in the cell below 64 and move on to the next column. This time, $39>32$, so we write a \textbf{1} below 32 and the new value to evaluate against is $39-32=7$. Again, $32<8<16$, so the columns of 16 and 8 both get a \textbf{0}. Since $7>4$, it gets a \textbf{1}, new value is $7-4=3$. This value, $3>2$, so the cell below 2 gets a value of \textbf{1} and the remainder is $3-2=1$. Finally, since 1 is the remainder from the last step, the cell for $1 == 1$ and hence, the last cell gets a 1 as well. Then, we have the equivalence $(167)_{10} = (10100111)_2$.

\subsection{Exercises}
\subsubsection{i) Convert 01101011 to decimal}
\textbf{Solution} - First we draw the binary conversion table. Then, we add all the cells that have a \textbf{1} below them and ignore the ones associated with a \textit{0}.

\begin{table}[H]
	\centering
	\begin{tabular}{cccccccc}
		\toprule
		$2^7$ &$2^6$ &$2^5$ &$2^4$ &$2^3$ &$2^2$ &$2^1$ &$2^0$ \\
		\textbf{128} &\textbf{64} &\textbf{32} &\textbf{16} &\textbf{8} &\textbf{4} &\textbf{2} &\textbf{1} \\
		\midrule
		0 &1 &1 &0 &1 &0 &1 &1 \\
		\bottomrule
	\end{tabular}
	\vspace{-5pt}
	\caption{Binary Conversion Table}
\end{table}
\vspace{-10pt}
\noindent
Thus, we have the decimal value:
\begin{align*}
	d &= 64 + 32 + 8 + 2 + 1\\
	&= 107
\end{align*}
\noindent
So, we have the equivalence: $(1101011)_2 = (107)_{10}$ (\textbf{Ans}).

\subsubsection{ii) Convert 49 to binary}
\textbf{Solution} - We have the table:
\begin{table}[H]
	\centering
	\begin{tabular}{rcccccccc}
		\toprule
		&$2^7$ &$2^6$ &$2^5$ &$2^4$ &$2^3$ &$2^2$ &$2^1$ &$2^0$ \\
		&\textbf{128} &\textbf{64} &\textbf{32} &\textbf{16} &\textbf{8} &\textbf{4} &\textbf{2} &\textbf{1} \\
		\midrule
		\textbf{Quotient}  &0  &0  &1  &1 &0 &0 &0 &1 \\
		\midrule
		\textbf{Remainder} &49 &49 &17 &1 &1 &1 &1 &0 \\
		\bottomrule
	\end{tabular}
	\vspace{-5pt}
	\caption{Decimal 49 to Binary}
\end{table}
\vspace{-10pt}
\noindent
So, we find that $(49)_{10} = (110001)_2$. (\textbf{Ans}).

\section{IPv4 Address Formatting}
An IPv4 address can be divided into two parts: the Host ID and the Network ID. The Host ID is the address of the device (PC, laptop, smartphone, printer, etc.) on the network. These are represented by the \textbf{host bits} of the IP address. The remaining bits for the Network ID and represent the network itself. Somewhere among those 32 bits, is a division that separates the Network ID from the Host ID, dictated by the \textbf{subnet mask}. 

Since the number of bits in the IPv4 Address is constant, i.e., 32 - if we need to represent more devices in a network, we need more bits, which leaves less room for network address bits. If however, we need to represent a large number of network with few devices in each network (called \textbf{subnets}), then we need a larger number of bits for network ID and fewer in the host bits. The larger the size of the network (i.e., the number of devices per network) the fewer networks (i.e., subnets) there can be, and vice versa. 

\subsection{Subnet Mask}
A \textbf{subnet mask} is used to figure out which bits in the IP address are parts of the network ID and which bits are the host ID. Thus, a subnet mask tells us about the size of each subnet (\textit{based on the number of bits in the host id}) as well as the total number of subnets that there can be (\textit{based on the number of bits in the network ID}). 

An IP address is much like an address in the real world, where neighbours living on the same street share parts of the address (street name, city, country, etc.) but have different house numbers. So, the Network ID is like the street name, i.e., shared by all the hosts in the network, but the Host ID is like the house number, unique to every host. All of the devices in a network share a common network address space.

A subnet mask is a set of contiguous \textbf{1}s followed by a contiguous set of \textbf{0}s (i.e., a subnet mask always has a series of 1s followed by a series of 0s, and never mixes them).

\begin{center}
	\begin{tabular}{rcccc}
		\toprule
		\textbf{Dotted Decimal Notation} &10 &1 &2 &3 \\
		\textbf{Subnet Mask in DDN} &255 &0 &0 &0 \\
		\midrule
		\textbf{IP in Binary}	&00001010 &00000001 &00000010 &00000011\\
		\textbf{Subnet Mask} &11111111 &00000000 &00000000 &00000000\\
		\midrule
		&\textit{Network Bits} &\multicolumn{3}{c}{|<--------------- \textit{Host Bits} --------------->|} \\ 
		\textbf{Network ID in Binary ($\wedge$)} &00001010 &00000000 &00000000 &00000000 \\
		\midrule
		\textbf{Network ID in DDN} &10 &0 &0 &0 \\
		\textbf{Host ID in DDN} & &1 &2 &3 \\
		\bottomrule
	\end{tabular}
\end{center}

\subsection{Prefix/Slash Notation of Subnet Mask}
Instead of writing the subnet mask in DDN, we can also just say the number of bits in the Network ID by writing the IP address as \textbf{(IP address)/(number of bits in Network ID)}. So, the above case, where the IP address is \verb|10.1.2.3| and the number of bits in the network ID is \verb|8|, the resultant \textbf{prefix notation} would be \verb|10.1.2.3/8|. 

\subsection{Dotted Decimal Notation of Subnet Mask}
We also have the option of writing the subnet mask in DDN, just like an IP address. Since the subnet mask consists of the first 8 bits, i.e., the first octet, the value of the subnet mask is \textbf{255} for the first octet (i.e., the maximum possible value of an octet) followed by 0s. Thus, the subnet mask in DDN is: \verb|255.0.0.0|.

\noindent

\begin{table}[H]
	\centering
	\begin{tabular}{rl}
		\toprule
		\textbf{Terms} &\textbf{Description} \\
		\midrule
		\textbf{IP address with no subnet information:}	&\verb|10.1.2.3|\\
		\textbf{IP Address in Prefix Notation}	&\verb|10.1.2.3/8|\\
		\textbf{IP Address in Dotted Decimal Notation}	&\verb|10.1.2.3 255.0.0.0|\\
		\bottomrule
	\end{tabular}
	\vspace{-5pt}
	\caption{Types of Subnet Mask representations}.
\end{table}
\vspace{-10pt}

\subsection{Calculating Network and Host ID}
The network ID of a particular network is obtained by taking the original IP address and changing all the host bits to \textbf{0}s. Thus, the network address/ID for the given IP address is \verb|10.0.0.0/8| in prefix notation, or \verb|10.0.0.0 255.0.0.0| in DDN. 

\section{Address Classes}
When we look at an IP address, we don't know which of the 32-bits represent the Network address and which represent the host ID unless we're provided with the Subnet Mask. While we can set the subnet mask ourselves, there are defaults, on the basis of which the entire IPv4 Address space (\verb|1.0.0.0 - 255.255.255.254|) can be divided into 5 classes: Classes A, B, C, D \& E. 

\begin{table}[H]
	\centering
	\begin{tabular}{rcll}
		\toprule
		\textbf{Address Class} &\textbf{Value in First Octet} &\textbf{Classful Mask (DDN)} &\textbf{Classful Mask (Prefix)}\\
		\midrule
		A &1 - 126 &255.0.0.0 &/8 \\
		B &128 - 191 &255.255.0.0 &/16 \\
		C &192 - 223 &255.255.255.0 &/24 \\
		D &224 - 239 &N/A &N/A \\
		E &240 - 255 &N/A &N/A \\
		\bottomrule
	\end{tabular}
	\vspace{-5pt}
	\caption{Classes of IPv4 Addresses}.
\end{table}
\vspace{-20pt}

\noindent
Among these, only classes A, B \& C are ever assigned to a normal host. Address \textbf{class D} is used for \textbf{Multicasts}, but even in that we send data to a multicast address, but no host is ever assigned an address in this range. It's a destination only address, and thus doesn't even have a default subnet mask. A \textbf{Class E} address is used for \textit{experimental purposes and R\&D}. 

The \textbf{Classful Mask}, also called the \textit{Natural Mask} is the default subnet mask for a specific address class. So, if an address such as \verb|10.1.2.3| uses a \textit{classful mask}, we know it's a Class A address (since first octet is in the range \textit{1 - 126}) which has a default mask of \verb|255.0.0.0| and so - \verb|10.0.0.0| is the network ID and \verb|.1.2.3| is the host ID.

The class of an IP address is entirely dependent upon the first octet. It doesn't matter what the subnet mask is, only the first octet determines the class of an IP address. 

\subsection{Loopback Addresses}
The \verb|127.0.0.0| network is entirely missing from the IP class table, because the entire network is used for a special purpose.  It's called the \textbf{Loopback address}, and it's used to test the functionality of the TCP/IP stack on a host machine, by pinging \textbf{127.0.0.1}. Thus, it's also called the local address of a device. 

\subsection{Assigning IP Addresses}
Publicly routable IP addresses are assigned by \textbf{ICANN (\textit{Internet Corporation for Assigned Names and Numbers})}. It's a non-profit organization that \textit{gives} a block of numbers for registration to an \textit{international registry}, such as \textbf{ARIN (\textit{American Registry for Internet Numbers})} in North America and \textbf{IANA (\textit{Internet Assigned Numbers Authority})} outside North America, operated by ICANN. 

\section{Private vs. Public IPv4 Addresses}
There can be a total of $2^{32} = 4,294,967,296$ IPv4 Addresses, and the total number of internet connected devices has already exceeded that. Thus, there are not enough IPv4 addresses to identify each device uniquely, a problem which was solved by the use of IPv6. However, since there's a sizeable IPv4 deployment in the industry, a lot of companies use a \textbf{Private IP address range} for their hundreds/thousands of devices that are used internally. 

These addresses are also called \textbf{RFC 1918} addresses. These addresses aren't publicly routable, i.e., no host on the internet should have one of these as their IP address. These exist solely to provide routing within a company/organization. These private IP addresses need to be translated to a public IP addresses when communicating with the internet (so that other computers can send data to them), and multiple private IP addresses can have the same public IP address. 

For example, using a wireless router at home, one can have multiple devices connected to the internet at the same time, using the same external IP address. This is possible because of \textbf{NAT \textit{(Network Address Translation)}}, a way to convert private IP addresses to publicly routable IP addresses. The private IP address space as defined in RFC 1918 are:

\begin{table}[H]
	\centering
	\begin{tabular}{rlll}
		\toprule
		\textbf{Class} &\textbf{Addr Block} &\textbf{Address Range} &\textbf{Default Subnet Mask} \\
		\midrule
		A &10.0.0.0/8 &10.0.0.0 - 10.255.255.255 &255.0.0.0 (/8) \\
		B &172.16.0.0/12 &172.16.0.0 - 172.31.255.255 &255.255.0.0 (/16) \\
		C &192.168.0.0/16 &192.168.0.0 - 192.168.255.255 &255.255.255.0 (/24) \\
		\bottomrule
	\end{tabular}
	\vspace{-5pt}
	\caption{Private IPv4 Address Ranges}.
\end{table}
\vspace{-20pt}

\subsection{Automatic Private IP Addressing (APIPA)}
In addition to the RFC 1918 private IP ranges, there is another private IP address range, but one that's not routable - even by any router that we own, because this IP address range is self-assigned. Generally, we don't want a device in our network to have an IP within this range. When a network connected device isn't assigned an IP address, either manually or automatically from a \textbf{DHCP (\textit{Dynamic Host Configuration Protocol})} server, it self-assigns an IP address to itself from this range. Such an address is called an \textbf{APIPA (\textit{Automatic Private IP Addressing})} address. 
\begin{table}[H]
	\centering
	\begin{tabular}{rlll}
		\toprule
		\textbf{Class} &\textbf{Addr Block} &\textbf{Address Range} &\textbf{Default Subnet Mask} \\
		\midrule
		B &169.254.0.0/16 &169.254.0.0 - 169.254.255.255 &255.255.0.0 (/16) \\
		\bottomrule
	\end{tabular}
	\vspace{-5pt}
	\caption{APIPA Address Range}.
\end{table}
\vspace{-20pt}

\noindent
A host having an APIPA address is an indication that something on the network has failed and this host needs attention because it's not connected to the rest of the network. Another case when APIPA is used is when two or more hosts have to communicate within themselves but there's no IP assignment, such as the case where two hosts are connected by a network cable but no further configuration. 

\section{Unicast} 
Unicast, Broadcast and Multicast are some of the different traffic flows in IPv4 networks. \textbf{Unicast} is a \textit{one-to-one} communication flow, i.e., one device on the network is talking to another device on the network. For example, if a host sends some query to a server, and the server responds with the results only to that host, it's unicast. 

\begin{figure}[H]
	\centering
	\includegraphics[width=0.7\linewidth]{"Mod1/chapters/1.6.a Unicast"}
	\caption{Unicast}
	\label{fig:unicast}
\end{figure}
\vspace{-15pt}

\noindent
Let us consider the above situation where the hosts with host IDs \verb|.1| and \verb|.2| want to receive a video, but the host \verb|.3| doesn't. Then the video server has to send two packets - one two \verb|10.1.1.1| and another to \verb|10.1.1.2| when using an unicast model. While this is good for host \verb|10.01.1.3|, since it didn't have to see unwanted traffic, the video server at \verb|10.1.1.100| has an increased load. It has to send multiple copies of the same data to multiple hosts individually. If there had been 200 hosts that wanted the video, then it'd have to send 200 individual copies, causing more processor load for the video server and bandwidth demand on the link connecting to the video server. 

\section{Broadcast}
A \textbf{broadcast} is a \textit{one-to-all} communication flow. Thus, the communication occurs from one source to all the devices in a subnet. 

\begin{figure}[H]
	\centering
	\includegraphics[width=0.7\linewidth]{"Mod1/chapters/1.6.b Broadcast"}
	\caption{Broadcast}
	\label{fig:broadcast}
\end{figure}

\noindent
In the above case, the video server sends the data to all the connected hosts, including PC \#3, i.e., \verb|10.1.1.3|. So, \verb|.3|'s NIC had to take some time to analyse the packet and then discard it, since it didn't request for that data. This will be done by PC \#3 for every single packet in the video stream that's being broadcast. 

This solved the problem of processor load of the video server, since it only has to broadcast one packet and bandwidth usage on the link connecting the video server, since only one packet flows through it. However, now PC \#3 has to keep rejecting the packets of the video. 

IPv4 utilizes broadcast a lot. The early versions of the \textbf{Routing Information Protocol (RIP)} used to advertise network information through broadcasts. Broadcasts can also be used to advertise services such as printer services on a local subnet. However, IPv6 doesn't use a broadcast at all!

\section{IPv4 Multicast}
\textbf{Multicast} is a \textit{one-to-many} communication flow. In multicast, the devices that want to receive the data join a \textbf{multicast group}. The multicast group has an IP address in the \textbf{Class D} range, i.e., \verb|224.0.0.1 - 239.255.255.254|. If the multicast group address is, say, \verb|239.1.1.1|, then it represents every host within that group. 

\begin{figure}[H]
	\centering
	\includegraphics[width=0.7\linewidth]{"Mod1/chapters/1.6.c Multicast"}
	\caption{Multicast}
	\label{fig:multicast}
\end{figure}
\vspace{-15pt}

\noindent
So now, the video server can send out one packet with the destination address of \verb|239.1.1.1| and it'll reach all the hosts within that group who want to receive the video, while those that don't want to, don't get any packets at all!

Multicasts are used quite a bit in both IPv4 as well as IPv6. In IPv4, many routing protocols such as RIPv2, OSPF, EIGRP, etc., use multicasting to efficiently send out network advertisements. A router can only send out network advertisements to the other routers in a multicast group. 

\section{The Need for Subnetting}


\section{Calculating Available Subnets}
\section{Calculating Available Hosts}
\section{Subnetting Practice Exercise \#1}
\section{Subnetting Practice Exercise \#2}
\section{Subnetting Practice Exercise \#3}
\section{Calculating Usable Ranges of IPv4 Addresses}
\section{Subnetting Practice Exercise \#4}
\section{Subnetting Practice Exercise \#5}
\section{Classless Inter-Domain Routing (CIDR)}
