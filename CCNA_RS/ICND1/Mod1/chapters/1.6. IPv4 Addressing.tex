\chapter{IPv4 Addressing}
Even though the global pool of IPv4 addresses has been exhausted, and IPv6 is the future, we still have a lot of IPv4 addressing in networking today. Thus, we need the following basic skills: 
\vspace{-10pt}
\begin{itemize}
	\item Calculating the number of subnets for a subnet mask.
	\item Calculating the number of hosts for a subnet mask.
	\item Calculating the usable address range for a specific subnet.
	\item Finding the Broadcast ID, Host IDs and Network ID in a network, etc. 
\end{itemize}

\section{Binary Numbering}
An example of an IPv4 Address (or simply, an IP address) is \textbf{10.1.2.3}. Before we can start working with IP addresses and use them to design networks, and calculate the specifics of a network, we need to be able to convert an IP address into its binary equivalent. An IPv4 address is 32-bits, and we use the \textbf{Dotted Decimal notation} to represent it, dividing the 32 bits into four sets of 8 bits each, called an \textbf{octet} and then writing the decimal equivalent of each octet, separated by dots, to represent the IP address. 

\begin{center}
	\begin{tabular}{rcccc}
		\toprule
		\textbf{Dotted Decimal Notation} &10 &1 &2 &3 \\
		\midrule
		\textbf{Binary Equivalent}	&1010 &1 &10 &11\\
		\textbf{8-bit Binary (Octet)}	&00001010 &00000001 &00000010 &00000011\\
		\textbf{Octet Number} &$1^{st}$ Octet &$2^{nd}$ Octet &$3^{rd}$ Octet &$4^{th}$ Octet  \\
		\bottomrule
	\end{tabular}
\end{center}

\subsection{Binary to Decimal Conversion}
Since binary numbers are \textbf{base-2} like decimal numbers are \textbf{base-10}, we need a binary conversion table such as the one shown below to convert a binary number to it's decimal equivalent.

\begin{table}[H]
	\centering
	\begin{tabular}{cccccccc}
		\toprule
		$2^7$ &$2^6$ &$2^5$ &$2^4$ &$2^3$ &$2^2$ &$2^1$ &$2^0$ \\
		\textbf{128} &\textbf{64} &\textbf{32} &\textbf{16} &\textbf{8} &\textbf{4} &\textbf{2} &\textbf{1} \\
		\bottomrule
	\end{tabular}
	\vspace{-5pt}
	\caption{Binary Conversion Table}
\end{table}
\vspace{-10pt}

\noindent
The above table simplifies the formula: 
\begin{equation}
	\text{Decimal} D = \sum_{i=n}^{0} 2^n b_i 
	\label{eqn:bin2dec}
\end{equation}
\noindent where $b_i$ is the binary bit in the $i^{th}$ position of the binary number $B$, $n$ is the number of digits in $B$ and $D$ is the decimal representation of $B$. 

\subsubsection{Converting 10010110 to decimal}
First we construct the binary conversion table and write the binary digits beneath the powers of 2. 

\begin{table}[H]
	\centering
	\begin{tabular}{cccccccc}
		\toprule
		$2^7$ &$2^6$ &$2^5$ &$2^4$ &$2^3$ &$2^2$ &$2^1$ &$2^0$ \\
		\textbf{128} &\textbf{64} &\textbf{32} &\textbf{16} &\textbf{8} &\textbf{4} &\textbf{2} &\textbf{1} \\
		\midrule
		1 &0 &0 &1 &0 &1 &1 &0 \\
		\bottomrule
	\end{tabular}
	\vspace{-5pt}
	\caption{Binary Conversion Table}
\end{table}
\vspace{-10pt}

\noindent
Now, from the above table, we simply add the powers of 2 (column headings) that have a 1 below them, and ignore those that have a zero in the binary representation to get the decimal equivalent.
\begin{align*}
	d &= 128 + 16 + 4 + 2\\
	  &= 150
\end{align*}
So, the binary number $(10010110)_2$ in decimal is $(150)_{10}$. This can be directly calculated with the equation \ref{eqn:bin2dec}: 
\begin{align*}
	d &= \sum_{i=7}^{0} 2^i b_i \\
	&= (2^7\times1) + (2^6\times0) + (2^5\times0) + (2^4\times1) + (2^3\times0) + (2^2\times1) +( 2^1\times1) + (2^0\times0) \\
	&= 128 + 16 + 4 + 2\\
	&= 150
\end{align*}

\section{IPv4 Address Formatting}
\section{Address Clases}
\section{Private vs. Public IPv4 Addresses}
\section{Unicast}
\section{Broadcast}
\section{IPv4 Multicast}
\section{The Need for Subnetting}
\section{Calculating Available Subnets}
\section{Calculating Available Hosts}
\section{Subnetting Practice Exercise \#1}
\section{Subnetting Practice Exercise \#2}
\section{Subnetting Practice Exercise \#3}
\section{Calculating Usable Ranges of IPv4 Addresses}
\section{Subnetting Practice Exercise \#4}
\section{Subnetting Practice Exercise \#5}
\section{Classless Inter-Domain Routing (CIDR)}
