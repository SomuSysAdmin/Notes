\chapter{Virtual LANs (VLANs)}

\section{Virtual LANs (VLANs) : Introduction}
When interconnecting devices on a switch, we may want some of those devices to be on different logical networks or subnets. This is done by assigning them to a different \textbf{VLAN (\textit{Virtual Local Area Network})}. 

For example, in an office building, there may be a dedicated VLAN for each department, i.e., a VLAN for the accounts team and one for the sales team so that the data for one department doesn't flow through the network of another team. We can assign different switch-ports to different VLANs. When we want traffic to flow between those VLANs, a router has to be used to route traffic from one subnet to another. 

\section{VLAN Theory}
Let us consider a case where we have two floors in an office which has 2 departments: Sales and Accounting. Since we don't want traffic to intermix, we need separate switches on each floor on each department. Given that we have 2 floors with 2 department each, the total number of switches is 4. For a bigger office, with more floors and department each, we'd need many more devices. 

\begin{figure}[H]
	\centering
	\includegraphics[width=0.7\linewidth]{"ICND1/1. Switches/chapters/2.2.a Without VLANs"}
	\caption{Without VLANs}
	\label{fig:2}
\end{figure}

\noindent
Further, for a new department to be added, a new switch has to be installed on each floor for that department. A solution to this problem is given by the introduction of VLANs. While the devices are connected to the same physical network, they devices themselves are connected to different logical networks. Thus, instantly the total number of switches required falls dramatically. 

\begin{figure}[H]
	\centering
	\includegraphics[width=0.7\linewidth]{"ICND1/1. Switches/chapters/2.2.b With VLANs"}
	\caption{With VLANs}
	\label{fig:2.2.b}
\end{figure}

\noindent
In either of the above cases, however, when data has to pass between floors, for example, when routing data between the Accounts and Sales subnets/broadcast domains/VLANs, there has to be one port dedicated on every switch and the router for each VLAN. However, if we use trunking, then these multiple ports can be combined into a single physical port that can carry data for multiple VLANs, while still maintaining separation. 

The primary reason we may want different VLANs for different networks is because it allows us to divide the broadcast domains, which means that performance is instantly increased. Further, devices won't be able to perform a packet capture of unknown broadcast/multicast/unicast packets.

\subsection{Packet Flow between VLANs}
Let us consider the situation in diagram \ref{fig:2.2.c}. We have PC$_1$ connected to VLAN$_{100}$, and PC$_2$ connected to VLAN$_{200}$, while the router is connected to the switch on a separate port, outside both VLANs, as a trunk (i.e., can carry data for multiple VLANs). 

\begin{figure}[H]
	\centering
	\includegraphics[width=0.9\linewidth]{"ICND1/1. Switches/chapters/2.2.c Router on a stick"}
	\caption{Routing between VLANs}
	\label{fig:2.2.c}
\end{figure}

\noindent
When data has to go from VLAN$_{100}$ to VLAN$_{200}$, they can't directly pass since they're on both different subnets in different VLANs. Thus, a router is required to convey traffic from one VLAN to another. 

So, data will go from the \textbf{ingress port} in VLAN$_{100}$, across the \textit{switching fabric}, i.e., switching back-plane to the router through the trunk port. The router will discover that this packet is destined for the network of VLAN$_{200}$ (\verb|192.168.1.0/24|), back across the trunk and then the switching fabric to reach the \textbf{egress port} in VLAN$_{200}$. 

This kind of a configuration where a router is connected to a switch to manage data across VLANs is called \textit{router on a stick} or \textbf{router on a trunk} connection. Some of the new Catalyst switches are Multilayer/Layer-3 Switches which don't need a router to perform the routing between VLANs. 

\section{VLAN Creation}
\subsection{Show existing VLANs}
To see a brief overview of the existing VLANs, we use the \verb|show vlan brief| command: 

\vspace{-15pt}
\begin{minted}{console}
sw1#sh vlan brief

VLAN Name                             Status    Ports
---- -------------------------------- --------- -------------------------------
1    default                          active    Gi0/0, Gi0/1, Gi0/2, Gi0/3
1002 fddi-default                     act/unsup
1003 token-ring-default               act/unsup
1004 fddinet-default                  act/unsup
1005 trnet-default                    act/unsup
\end{minted}
\vspace{-10pt}

\noindent
The fddi, token ring VLANs can be ignored. The main VLAN is the default, VLAN$_1$. 

\subsection{Creating a new VLAN}
Let us say we want a couple of new VLANs - VLAN$_{100}$ for the accounts department and VLAn$_{200}$ for the sales department in an office. Just like the interface configuration mode, we also have a VLAN configuration mode. The command used to name a VLAN is \verb|name|. We name the two VLANs using:

\vspace{-15pt}
\begin{minted}{console}
sw1(config)#vlan 100
sw1(config-vlan)#name ACCT
sw1(config-vlan)#exit
sw1(config)#vlan 200
sw1(config-vlan)#name SALES
sw1(config-vlan)#end
sw1#sh vlan br

VLAN Name                             Status    Ports
---- -------------------------------- --------- -------------------------------
1    default                          active    Gi0/0, Gi0/1, Gi0/2, Gi0/3
100  ACCT                             active
200  SALES                            active
...
\end{minted}
\vspace{-10pt}

\noindent
We see that the VLANs have been created and now we can assign ports to it.

\subsection{Deleting a VLAN}
To delete a VLAN, we use the command \verb|no vlan| followed by the VLAN number:

\vspace{-15pt}
\begin{minted}{console}
sw1(config)#vlan 300
sw1(config-vlan)#name TEST
sw1(config-vlan)#end
sw1#sh vlan br | i TEST
300  TEST                             active
sw1#conf t
Enter configuration commands, one per line.  End with CNTL/Z.
sw1(config)#no vlan 300
sw1(config)#end
sw1#sh vlan br | i TEST
sw1#
\end{minted}
\vspace{-10pt}

\noindent
Now, we may assume that the VLANs have been wiped, however, the information about VLANs are stored in a separate section of the memory, called the \textit{flash}. We can view its contents using \verb|show flash|. In it, we have a file called \verb|vlan.dat|. To truly erase all the VLANs, we need to use the \verb|delete flash:vlan.dat| command. 

\section{Assigning Ports to a VLAN}
When a port is dedicated to a single VLAN, it's called an \textbf{access} port. Contrastingly, there can be \textbf{trunk} ports carrying data over multiple VLANs over a single interface. To assign a port to a VLAN, we need to declare it as an access port using the \verb|switchport access| command. Let us consider we want to add the interface \textit{gi0/0} to VLAN 100. We use:

\vspace{-15pt}
\begin{minted}{console}
sw1(config)#int gi0/0
sw1(config-if)#switchport access vlan 100
\end{minted}
\vspace{-10pt}

\noindent
To put a range of interfaces all in the same VLAN, we use the \textit{interface range} configuration mode instead of teh \textit{interface} config mode. To put all ports between Gi0/1 and Gi0/3 in VLAN 300, we use: 

\vspace{-15pt}
\begin{minted}{console}
sw1(config)#interface range gi0/1 - 2
sw1(config-if-range)#switchport access vlan 200
sw1(config-if-range)#end
sw1#sh vlan br

VLAN Name                             Status    Ports
---- -------------------------------- --------- -------------------------------
1    default                          active    Gi0/3
100  ACCT                             active    Gi0/0
200  SALES                            active    Gi0/1, Gi0/2
...
\end{minted}
\vspace{-10pt}
