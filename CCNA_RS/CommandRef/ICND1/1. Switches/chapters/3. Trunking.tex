\chapter{Trunking}

\section{Trunking : Introduction}
Since we don't want to mix the traffic between the VLANs, and since they're in different subnets, if we have to send traffic between multiple switches in our network, it might seem like we need separate interfaces to carry the traffic for each VLAN, but that's not the case, since we can use trunk ports. Trunk interfaces allow us to send the data for multiple VLANs over a single connection. 

\section{Trunking Theory}
\begin{figure}[H]
	\centering
	\includegraphics[width=0.7\linewidth]{"ICND1/1. Switches/chapters/2.2.b With VLANs"}
	\caption{Using Access Ports}
\end{figure}

\noindent
Using trunks, we can combine the data flowing over both wires between switches, and between the switch and the router. 

\begin{figure}[H]
	\centering
	\includegraphics[width=0.7\linewidth]{"ICND1/1. Switches/chapters/3.2.a Trunking"}
	\caption{Using Trunk Ports}
	\label{fig:3.2.a}
\end{figure}

\noindent
This raises a question of how the switches determine what VLAN the frame belongs to. This is done by '\textit{tagging}' each frame (with a number) when it travels through the trunk as belonging to a certain VLAN. 

If a frame originating from the Accounting VLAN on the 2nd floor wants to go to the PC on the 1st floor, then the switch on the 2nd floor will tag the frame and when the switch on floor 1 gets it, it'll see that it's destined for the Accounts VLAN and send the frame to the appropriate port. 

In case of inter-VLAN communication, the originating switch tags the frame for the accounts VLAN while travelling to the router. The router then receives the frame on one sub-interface (\textbf{logical interface}) and sends it out through another sub-interface. When the router forwards the packet to the Sales VLAN, it'll change the tagging to that of Sales VLAN and send it back up the trunk, to the appropriate port for the Sales PC. 

\subsection{Marking Frames for VLANs}
For VLAN usage, 4 Bytes of data are added to the frame in accordance to the IEEE 802.1q frame format. Among these 4 bytes, 12 bits identify the particular VLAN a packet belongs to.  3 bits set the priority or \textbf{Quality of Service} for that frame. These 4B are: 

\begin{figure}[H]
	\centering
	\includegraphics[width=1\linewidth]{"ICND1/1. Switches/chapters/3.2.b VLAN Frame"}
	\caption{802.1q Frame}
	\label{fig:3.2.b}
\end{figure}

\subsection{Native VLAN}
The frames for a Native VLAN aren't tagged. This is the only VLAN that sends untagged data, and the native VLAN can be configured for a switch. Of course, this means that the native VLAN on both switches need to be the same, i.e., they must agree which VLAN will send untagged data. 

If one switch has native VLAN set to \textit{VLAN$_{100}$} while a switch on the other end of the connection has the native VLAN set to \textit{VLAN$_{200}$},  then when the first switch will send data destined for \textit{VLAN$_{100}$}, the second switch will accidentally forward it to the wrong VLAN, i.e., \textit{VLAN$_{200}$}. 

\section{Trunking Modes}
When two switches are interconnected with a trunk, there are different trunking modes that they may be arranged in. They are: 

\noindent
\begin{center}
	\begin{tabular}{rm{0.73\textwidth}}
		\toprule
		\textbf{Mode} &\textbf{Description} \\
		\midrule
		\textbf{Access}	&Forces a port to act as an Access port (i.e., unless it's for a voice VLAN, it will bear traffic for a single VLAN)\\
		\textbf{Trunk}	&Forces a port to act as a Trunk port\\
		\textbf{Dynamic Desirable}	&Initiates the negotiation of a trunk by sending the required \textbf{DTP} frames\\
		\textbf{Dynamic Auto}	&Passively waits for the remote switch to start negotiating a trunk formation.\\
		\bottomrule
	\end{tabular}
\end{center}

\subsection{Formation of a Trunk}
A trunk is formed via negotiation of the two switch ports at the end of a link through the \textbf{Dynamic Trunking Protocol (\textit{DTP})}, which is Cisco's proprietary protocol for this purpose. When a switch port set to \textbf{Dynamic Desirable} or \textbf{Dynamic Auto} mode receives a DTP frame, it will form a trunk. However, only a \textbf{Dynamic Desirable} or \textbf{Trunk} switch port can send the necessary DTP frames for trunk negotiation. Thus, we get the following table for trunk formations:

\noindent
\begin{center}
	\begin{tabular}{rrl}
		\toprule
		\textbf{Switch 1 Mode} &\textbf{Switch 2 Mode}&\textbf{Trunk Formation} \\
		\midrule
		Access &\textit{ANY} &\textbf{No}\\
		Trunk &Dynamic Desirable &\textit{Yes}\\
		Trunk &Dynamic Auto &\textit{Yes}\\
		Trunk &Trunk &\textit{Yes}\\
		Dynamic Desirable &Dynamic Desirable &\textit{Yes}\\
		Dynamic Desirable &Dynamic Auto &\textit{Yes}\\
		Dynamic Auto &Dynamic Auto &\textbf{\textit{No}}\\
		\bottomrule
	\end{tabular}
\end{center}

\noindent
In case of Access modes, no matter what kind of frames the port receives, it'll ignore them since it's been hard-coded to be an access port. When there's a switch in trunk mode on one side, and a switch port in dynamic mode on the other side, they will form a trunk since both/one side respectively will send the DTP frames. In case of Trunk/Trunk configuration, both ports are independently told to form a trunk and hence both will send DTP frames and form a trunk. 

In the case of Dynamic desirable ports on both sides, a trunk will form since one or both sides will receive the DTP frames from the partner and form a trunk. Similarly when there's dynamic auto on one side and dynamic desirable on the other, the later will send the DTP frames for trunk formation and the former will accept, thus forming a trunk. But when there's switch ports with \textit{dynamic auto} trunk mode on both side, neither will start the negotiation for trunk formation, although they're willing to form a trunk. Due to this, a trunk \textbf{won't be formed}. 

\section{Creating Trunks}
\subsection{Finding out the current trunk mode of a switch port}
The \verb|show interfaces <interfaceName> switchport| command shows us the VLAN/Trunk details of a switch port, which can be written as \verb|sh int <interfaceNumber> sw| for short: 

\vspace{-15pt}
\begin{minted}{console}
sw1#show interfaces g0/0 switchport
Name: Gi0/0
Switchport: Enabled
Administrative Mode: dynamic auto
Operational Mode: static access
Administrative Trunking Encapsulation: negotiate
Operational Trunking Encapsulation: native
Negotiation of Trunking: On
Access Mode VLAN: 1 (default)
Trunking Native Mode VLAN: 1 (default)
Administrative Native VLAN tagging: enabled
Voice VLAN: none
Administrative private-vlan host-association: none
Administrative private-vlan mapping: none
Administrative private-vlan trunk native VLAN: none
Administrative private-vlan trunk Native VLAN tagging: enabled
Administrative private-vlan trunk encapsulation: dot1q
Administrative private-vlan trunk normal VLANs: none
Administrative private-vlan trunk associations: none
Administrative private-vlan trunk mappings: none
Operational private-vlan: none
Trunking VLANs Enabled: ALL
Pruning VLANs Enabled: 2-1001
Capture Mode Disabled
Capture VLANs Allowed: ALL

Protected: false
Appliance trust: none
\end{minted}
\vspace{-10pt}

\noindent
In the above section, line \#4 \verb|Administrative Mode: dynamic auto| tells us that it's set to dynamic auto trunking mode, but the next line, \verb|Operational Mode: static access| tells us that since the other side hasn't sent the required DTP frames, it's acting as an access port. 

Let us consider we have 3 switches, sw1, sw2 and sw3 arranged in the following topology. Each interface starting with \verb|gi| is a \verb|gigabitEthernet| port:

\begin{figure}[H]
	\centering
	\includegraphics[width=0.57\linewidth]{"ICND1/1. Switches/chapters/3.4.a Trunk Topology"}
	\caption{Trunk Topology}
	\label{fig:3.4.a}
\end{figure}

\noindent
Let us consider we first want to set up the trunks between the switches \textit{sw1} and \textit{sw2}, and another trunk between the switches \textit{sw1} and \textit{sw3}. The interface we're currently inspecting in figure \ref{fig:3.4.a} is \verb|gi0/0| on \textit{sw1}, i.e., the one connecting to \verb|gi0/0| on \textit{sw2}. 

We also see on line \#6 that if a trunk is formed, the trunking encapsulation will be negotiated, due to: \verb|Administrative Trunking Encapsulation: negotiate|. We might want to hard-code it to \textbf{dot1q}. 

Finally, if a trunk were to be formed, the native VLAN (the one that carries frames that are untagged) will be set to VLAN1 due to line \#10 : \verb|Trunking Native Mode VLAN: 1 (default)|. We should also know how to change this behaviour. 

\subsection{Showing currently existing Trunks}
The trunks that exist on all the interfaces currently can be seen with \verb|show interfaces trunk|, or \verb|sh int tr| for short. Since there are none currently, the result looks like: 

\vspace{-15pt}
\begin{minted}{console}
sw1#show interfaces trunk
sw1#
\end{minted}
\vspace{-10pt}

\noindent
When there are existing trunks, it looks like:

\vspace{-15pt}
\begin{minted}{console}

\end{minted}

\subsection{Chaning Switch port's trunking settings}
To change the trunk formation settings of a switch port, we use the \verb|switchport trunk| command in the interface configuration mode. It has several parameters that we can control: 

\vspace{-15pt}
\begin{minted}{console}
sw1(config-if)#switchport trunk ?
  allowed        Set allowed VLAN characteristics when interface is in trunking
                 mode
  encapsulation  Set trunking encapsulation when interface is in trunking mode
  native         Set trunking native characteristics when interface is in
                 trunking mode
  pruning        Set pruning VLAN characteristics when interface is in trunking
                 mode
\end{minted}
\vspace{-10pt}

\subsubsection{Changing Trunk Encapsulation}
The trunk encapsulation determines the format in which the trunk negotiation is performed. This has the possible values of: 

\vspace{-15pt}
\begin{minted}{console}
sw1(config-if)#switchport trunk encapsulation ?
  dot1q      Interface uses only 802.1q trunking encapsulation when trunking
  isl        Interface uses only ISL trunking encapsulation when trunking
  negotiate  Device will negotiate trunking encapsulation with peer on
             interface
\end{minted}
\vspace{-10pt}

\noindent
\textbf{ISL} is a proprietary encapsulation by Cisco that Cisco itself doesn't recommend any more. \textbf{dot1q} is the standard we want to use and \textit{negotiate} will have the switch port negotiate and set up the encapsulation based on the partner's encapsulation settings. 

We can set it to \textit{dot1q} by using the \verb|switchport trunk encapsulation dot1q| command, or \verb|sw tr en dot| for short:

\vspace{-15pt}
\begin{minted}{console}
sw1(config-if)#sw tr e dot
sw1(config-if)#
\end{minted}
\vspace{-10pt}

\subsubsection{Changing the native VLAN}
The native VLAN can also be changed in a similar fashion using the \verb|switchport trunk native vlan| command (\verb|sw tr nat vlan| for short):

\vspace{-15pt}
\begin{minted}{console}
sw1(config-if)#sw tr n vlan 100
sw1(config-if)#
\end{minted}
\vspace{-10pt}

\subsubsection{Changing the switchport trunking mode}
We can change the current \textit{dynamic auto} trunking mode to \textbf{dynamic desirable} using the \verb|switchport mode dynamic desirable| command (\verb|sw m dy des| for short):

\vspace{-15pt}
\begin{minted}{console}
sw1(config-if)#switchport mode ?
  access        Set trunking mode to ACCESS unconditionally
  dot1q-tunnel  set trunking mode to TUNNEL unconditionally
  dynamic       Set trunking mode to dynamically negotiate access or trunk mode
  private-vlan  Set private-vlan mode
  trunk         Set trunking mode to TRUNK unconditionally

sw1(config-if)#switchport mode dynamic ?
  auto       Set trunking mode dynamic negotiation parameter to AUTO
  desirable  Set trunking mode dynamic negotiation parameter to DESIRABLE

sw1(config-if)#switchport mode dynamic desirable
sw1(config-if)#
\end{minted}
\vspace{-10pt}

\subsection{Trunk settings mismatch}
The effect of incompatibility of trunk settings on both sides of the link is immediate errors. As such, the moment we set the trunk's native VLAN on sw1's \verb|gi0/0| to \textbf{VLAN100} while the one on sw2's \verb|gi0/0| is still set to the default of \textbf{VLAN1}, we get: 

\vspace{-15pt}
\begin{minted}{console}
sw1(config-if)#switchport mode dynamic desirable
sw1(config-if)#
*Nov 21 07:53:56.502: %SPANTREE-2-RECV_PVID_ERR: Received BPDU with inconsistent peer vlan id 1 on GigabitEthernet0/0 VLAN100.
*Nov 21 07:53:56.504: %SPANTREE-2-BLOCK_PVID_PEER: Blocking GigabitEthernet0/0 on VLAN0001. Inconsistent peer vlan.
*Nov 21 07:53:56.505: %SPANTREE-2-BLOCK_PVID_LOCAL: Blocking GigabitEthernet0/0 on VLAN0100. Inconsistent local vlan.
*Nov 21 07:54:25.670: %CDP-4-NATIVE_VLAN_MISMATCH: Native VLAN mismatch discovered on GigabitEthernet0/0 (100), with sw2 GigabitEthernet0/0 (1).
*Nov 21 07:55:18.671: %CDP-4-NATIVE_VLAN_MISMATCH: Native VLAN mismatch discovered on GigabitEthernet0/0 (100), with sw2 GigabitEthernet0/0 (1).
...
\end{minted}
\vspace{-10pt}

\noindent
This can be changed by setting up the partner's (\verb|sw2 gi0/0|) trunk config appropriately and match the native VLAN. On sw2, we have: 

\vspace{-15pt}
\begin{minted}{console}
sw2#sh int trunk

Port        Mode             Encapsulation  Status        Native vlan
Gi0/0       auto             n-802.1q       trunking      1
...
sw2#
*Nov 21 08:03:44.500: %CDP-4-NATIVE_VLAN_MISMATCH: Native VLAN mismatch discovered on GigabitEthernet0/0 (1), with sw1.somuvmnet.com GigabitEthernet0/0 (100).
sw2#sh int g0/0 sw
Name: Gi0/0
Switchport: Enabled
Administrative Mode: dynamic auto
Operational Mode: trunk
Administrative Trunking Encapsulation: negotiate
Operational Trunking Encapsulation: dot1q
Negotiation of Trunking: On
Access Mode VLAN: 1 (default)
Trunking Native Mode VLAN: 1 (default)
\end{minted}
\vspace{-10pt}

\noindent
We see that a trunk has formed, but there's a VLAN mismatch. We match the settings with: 

\vspace{-15pt}
\begin{minted}{console}
sw2#conf t
Enter configuration commands, one per line.  End with CNTL/Z.
sw2(config)#int g0/0
sw2(config-if)#sw t en dot
sw2(config-if)#sw t nat v 100
sw2(config-if)#end
sw2#sh int tr

Port        Mode             Encapsulation  Status        Native vlan
Gi0/0       auto             802.1q         trunking      100
...
sw2#
*Nov 21 08:08:07.859: %SPANTREE-2-UNBLOCK_CONSIST_PORT: Unblocking GigabitEthernet0/0 on VLAN0001. Port consistency restored.
sw2#
\end{minted}
\vspace{-10pt}

\subsection{Hard-coded Trunk mode}
Let us quickly configure \textit{sw3} with the following: 

\vspace{-15pt}
\begin{minted}{console}
sw3#conf t
Enter configuration commands, one per line.  End with CNTL/Z.
sw3(config)#int g0/0
sw3(config-if)#sw mo dyn auto
sw3(config-if)#sw tr encap dot
sw3(config-if)#sw tr native vlan 200
\end{minted}
\vspace{-10pt}

\noindent
So, the trunk mode is set to \textbf{dynamic auto}, encapsulation to \textbf{dot1q} and the native VLAN to 200. Now, when we create a similar profile in \textit{sw1} but set the mode to \textbf{trunk} instead of a dynamic type, and execute the \verb|sh int tr|, we get:

\vspace{-15pt}
\begin{minted}{console}
sw1#sh int tr

Port        Mode             Encapsulation  Status        Native vlan
Gi0/0       desirable        802.1q         trunking      100
Gi0/1       on               802.1q         trunking      200

Port        Vlans allowed on trunk
Gi0/0       1-4094
Gi0/1       1-4094

Port        Vlans allowed and active in management domain
Gi0/0       1,100,200
Gi0/1       1,100,200

Port        Vlans in spanning tree forwarding state and not pruned
Gi0/0       1,100,200
Gi0/1       1,100,200
\end{minted}
\vspace{-10pt}

\noindent
Note that the mode for \verb|sw1 gi0/1| states \textbf{on} instead of \textit{auto/desirable}. This indicates that the DTP frames are being sent by this particular interface to form a trunk and it's a static trunking. 

\section{VLAN pruning}
By default, all VLANs are allowed over all trunks. This can be bad for security, given that all unknown broadcast, multicast and unicast traffic is sent over trunks. Thus, it'd enable malicious attackers to snoop on the traffic flowing through the trunk. Further, limiting VLANs reduces bandwidth utilization on the trunk and thus helps from a \textbf{Quality of Service (\textit{QoS})} perspective. 

\subsection{Showing VLANs allowed on a trunk}
To see the particular VLANs allowed on a trunk, we use the \verb|show interfaces <int#> trunk|. To see allowed VLANs for all trunks, we use the normal version, \verb|sh int tr|:

\vspace{-15pt}
\begin{minted}{console}
sw1#sh interfaces g0/0 trunk

Port        Mode             Encapsulation  Status        Native vlan
Gi0/0       desirable        802.1q         trunking      100

Port        Vlans allowed on trunk
Gi0/0       1-4094

Port        Vlans allowed and active in management domain
Gi0/0       1,100,200

Port        Vlans in spanning tree forwarding state and not pruned
Gi0/0       1,100,200
\end{minted}
\vspace{-10pt}

\noindent
We can see that over Gi0/0, all VLANs are allowed (VLANs 1-4094).

\subsection{Allowing only certain VLANs over a certain trunk}
Let us consider in the last example, sw1 only wants the VLAN$_{100}$ over the interface \verb|Gi0/0| and the VLAN$_{200}$ over the interface \verb|Gi0/1|. To allow a VLAN over a switch port, we use: \verb|switchport trunk allowed vlan <vlan numbers: 1,2,3...>| in the interface config mode.

\vspace{-15pt}
\begin{minted}{console}
sw1(config)#int g0/0
sw1(config-if)#sw tr allowed vlan 1,100
sw1(config-if)#end
sw1#sh int g0/0 tr

Port        Mode             Encapsulation  Status        Native vlan
Gi0/0       desirable        802.1q         trunking      100

Port        Vlans allowed on trunk
Gi0/0       1,100

Port        Vlans allowed and active in management domain
Gi0/0       1,100

Port        Vlans in spanning tree forwarding state and not pruned
Gi0/0       1,100
\end{minted}
\vspace{-10pt}

\noindent
In the case we only wanted VLAN 100 to be banned, we could've used the command: \verb|sw tr allowed vlan except 100|.

\vspace{-15pt}
\begin{minted}{console}
sw1(config)#int g0/1
sw1(config-if)#sw tr allowed vlan except 100
sw1(config-if)#end
sw1#sh interfaces trunk

Port        Mode             Encapsulation  Status        Native vlan
Gi0/0       desirable        802.1q         trunking      100
Gi0/1       on               802.1q         trunking      200

Port        Vlans allowed on trunk
Gi0/0       100
Gi0/1       1-99,101-4094

Port        Vlans allowed and active in management domain
Gi0/0       100
Gi0/1       1,200

Port        Vlans in spanning tree forwarding state and not pruned
Gi0/0       100
Gi0/1       1
\end{minted}
\vspace{-10pt}